\documentclass{patmorin}
\usepackage{pat}
\usepackage{hyperref}
\usepackage{paralist}
\usepackage{graphicx}
\hypersetup{colorlinks=true, linkcolor=linkblue,  anchorcolor=linkblue,
citecolor=linkblue, filecolor=linkblue, menucolor=linkblue,
urlcolor=linkblue, pdfcreator=Me, pdfproducer=Me} 
\setlength{\parskip}{1ex}

\title{\MakeUppercase{Optimal Robust Spanners in One Dimension}}

\author{Prosenjit Bose, Paz Carmi, Michael van Dyk, Pat Morin, 
  \newline
  Luis Fernando Schulz Xavier da Silveira,
  and Jan Volec}

\newcommand{\note}[2]{{\color{red}[#1:~#2]}}

\begin{document}
\maketitle


\begin{abstract}
  For any $\epsilon >0$ and any $n\in\N$, we construct a graph $G=(V,E)$
  with vertex set $V=\{0,\ldots,n-1\}$ having $O(\epsilon^{-2}n\log n)$
  edges, and such that, for any $F\subseteq V$, $G-F$ contains a monotone
  path of length at least $n-(1+\epsilon)|F|$.  The number of edges
  in $G$ matches a lower-bound of Bose \etal\ (2013) and improves the
  previous construction of Buchin, Hulshof and Ol\'ah (2018) which has
  $O(n^{1+\epsilon})$ edges and guarantees $|F^*|\le c^{1/\epsilon}|F|$,
  for some constant $c\ge 3$ and any $\epsilon >0$.
\end{abstract}

\section{Introduction}

A geometric graph $G=(V,E)$ with vertex set $V\subset\R^d$ is a (geometric)
$t$-spanner of a subset $X\subset V$ if, for every pair of vertices
$v,w\in X$, $v\neq w$,
\begin{equation}
	\frac{\|vw\|_G}{\|vw\|} \enspace , \eqlabel{spanning-ratio}
\end{equation}
where $\|vw\|$ denotes the Euclidean distance between $v$ and $w$ and
$\|vw\|_G$ denote the Euclidean length of the shortest path between $v$
and $w$ in $G$, where use the convention that $\|vw\|_G=\infty$ if $v$
and $w$ are in different components of $G$.  
Most of the research on spanners focuses on \emph{sparse}
spanners, where the number of edges in $G$ is linear, or close to linear,
in $|V|$.  In addition to having natural applications to transportation
networks, sparse $t$-spanners have found numerous applications in
approximation algorithms and geometric data structures.

For any non-decreasing function $f\colon\N\to\N$, Bose \etal\ say
that a geometric graph $G$ is an \emph{$f(k)$-robust $t$-spanner}
if, for every set $F\subset V(G)$, there exists a set $F^*\supseteq
F$ such that $|F^*|\le f(|F|)$ and the graph $G-F$ is a $t$-spanner
of $V(G)\setminus F^*$.  In networking applications, this definition
captures the idea that the number of nodes harmed by a set faulty nodes
should be bounded by a function of the number of faulty nodes.

Under this definition, the most robust spanner one could hope for
would be a $k$-robust spanner, but it is straightforward to argue
that, even for one dimensional point sets, the complete graph is the
only $k$-robust spanner.\footnote{Proof: Consider any pair of vertices
$v,w\in V$ that are not adjacent in $G$ and let $F=V\setminus\{v,w\}$.
Then $\|vw\|_{G-F}=\infty$ so $G-F$ is a not a $t$-spanner of $V\setminus
F=V\setminus F^*$ for any $t<\infty$.} The complete graph is not sparse,
and is therefore not suitable for many applications. A natural second-best
option is a $ck$-robust spanner with a near-linear number of edges,
for some $c>1$ .

Bose \etal \cite[Theorem~3]{bose.dujmovic.ea:robust} show that for any
constants $c,t>1$ and any $n$-element set $V\subset\R$, any $ck$-robust
$t$-spanner $G=(V,E)$ must have $\Omega(n\log n)$ edges.  Their work
does not give any construction matching this lower bound, leaving a
gap in our understanding of robust spanners, even in one dimension.
Buchin, Hulshof and Ol\'ah \cite{buchin.hulshof.olah:robust} made the
first progress in filling this gap by constructing, for any $\delta>0$,
a $k2^{O(1/\delta)}$-robust spanner 1-spanner having $O(n^{1+\delta})$
edges.  In the current paper, we prove the following result:

\begin{thm}\thmlabel{main}
  For every $n$-element set $V\subset\R$ and every $\epsilon>0$,
  there exist a $(1+\epsilon)k$-robust 1-spanner $G=(V,E)$ with
  $O(\epsilon^{-2}n\log n)$ edges.
\end{thm}

The lower-bound discussed above implies that the number of edges
in \thmref{main} is optimal, except possibly for the dependence on
$\epsilon$.


\section{The Construction}


Without loss of generality, assume $n$ is a power of 2 and refer to
\figref{sets}.  Construct a complete ordered binary tree $T$ of height
$\log_2 n$ whose nodes are contiguous sets of integers.  The leaves of
$T$, in left-to-right order, are the singleton sets $\{0\},\ldots,\{n-1\}$
and each internal node of $T$ is the set obtained by taking the union
of its two children.  Stated another way, $T=T(0,\log_2 n-1)$, whree
$T(i,b)$ has a single node $\{i\}$ if $b=0$. Otherwise $T(i,b)$ has
a root node $\{i,i+1,\ldots,i+2^{b-1}-1\}$ whose two children are the
subtrees $T(i,b-1)$ and $T(i+2^{b-1}, b-1)$.

\begin{figure}
  \begin{center}
   \includegraphics[width=.98\textwidth]{figs/binary-tree-1}
  \end{center}
  \caption{The binary tree $T$ on $\{0,\ldots,n-1\}$.}
\end{figure}

Note that the nodes of $T$ can be partitioned in levels $L_0,\ldots,L_{\log_2 n}$, where each $L_i=\{u_{i,0},\ldots,u_{i,2^i-1}\}$ and
\[
	u_{i,k} = \{kn/2^i,\ldots,(k+1)n/2^{i}-1\} \enspace .
\]
For each consecutive pair $(u_{i,k},u_{i,k+1})$ of nodes in $L_i$ we say
that $u_{i,k}$ and $u_{i,k+1}$ are \emph{buddies} and we call $u_{i,k+1}$
the \emph{right buddy} of $u_{i,k}$.  For any $k\in\{0,\ldots,2^{i}-1\}$,
we use the notation $r(u_{i,k})$ to denote the right-buddy, $u_{i,k+1}$,
of $u_{i,k}$.

\subsection{Expanders}

Our construction $G$ will consist of the union of many \emph{expander graphs} whose properties are given by the following theorem:

\begin{defn}
   A bipartite graph $B=(X,Y,E)$ with $|X|=|Y|=n$ is called
   an \emph{order-$n$ $\epsilon$-expander} for $(X,Y)$ if, for
   every $X'\subseteq X$, $|N_B(X')| \ge \min\{(1-\epsilon/2)n,
   4|X'|/\epsilon\}$.
\end{defn}

It is known that, for any $\epsilon >0$ and any $n\in\N$, there exists
an order-$n$ $\epsilon$-expander having $O(\epsilon^{-1}n\log n)$
edges \cite{X}.

Our graph $G$ will contain many $\epsilon$-expanders for pairs $(u,w)$
where $u,w\in L_i$ are nodes at the same level, $i$, of $T$.  Given a
set $F\subseteq V$, we say that a node $u$ of $T$ is \emph{sparse}
with respect to $F$ if $|u\cap F| \le (1-\epsilon)|u|$ and $u$ is
\emph{dense} (with respect to $F$) otherwise.  In other words, if $u$
is sparse then $|u\setminus F| \ge \epsilon |u|$.

The following observation is a consequence of the definitions so far:

\begin{obs}\obslabel{extend}
	Let $n$, $G$, $F$, $T$, and $L_0,\ldots,L_{\log_2 n}$ be as defined above and
  and assume the following:
  \begin{compactenum}
     \item $x\in \{1,\ldots,n\}$ is a vertex and $u,w\in L_i$ 
      is a pair of nodes of $T$;
     \item $\{x\} \prec u\prec w$;
     \item $G$ contains an $\epsilon$-expander, $B_{uw}$, for $(u,w)$; and
     \item $u'\subseteq u$, $|u'|\ge\epsilon|u|/4$, and, for every $x'\in u'$, $G-F$
   contains a monotone path from $x$ to $x'$.
   \end{compactenum}
   Then there exists a
	subset $w'\subseteq w$, $|w'|\ge(1-\epsilon/2)|u|-|F\cap w|$, and, for
   every $y\in w'$, $G-F$ contains a monotone path from $x$ to $y$.
\end{obs}

\begin{proof}
  By the definition of $\epsilon$-expander, $N_{B_{uw}}(u')$ is a
  subset of $w$ of size at least $(1-\epsilon/2)|w|$. Therefore
  $w'=N_{B_{uw}}(u')\setminus F$ has size $|w'|\ge
  (1-\epsilon/2)|w|-|F\cap w|$.  Every vertex $y$ in $w'$ has a neighbour
  $x'\in u'$. Appending the edge $x'y$ to the monotone path from $x$
  to $x'$ gives a monotone path from $x$ to $y$.
\end{proof}
   
	
\subsection{A $(2+\epsilon)k$-Robust Construction}

We now describe a first construction of $G$ that has $O(\epsilon^{-1}n\log
n)$ edges and guarantees that, for any $F\subseteq V$,  $G-F$ has a
monotone path of length at least $n-(2+O(\epsilon))|F|$.  Changing
notation slightly, this means that, for any $\delta >0$, $G$ is a
$(2+\delta)k$-robust 1-spanner with $O(\delta^{-1}n\log n)$ edges.

The graph $G$ contains, for each pair $(u,w)$ of buddies in $T$, an
expander graph $B_{uw}$ for the pair $(u,w)$.  The graph $B_{uw}$ has $O(|u|/\epsilon)$ edges. Therefore, for each level $L_i$, the buddy pairs in $L_i$ contribute $O(n/\epsilon)$ edges to $G$.  Summing over all levels we see that $G$ has $O(\epsilon^{-1}n\log n)$ edges.

Next we define a set $F^+\supseteq F$ as follows: 
\[
	F^+ = \cup\{ u\in V(T): \text{$u$ is dense or $r(u)$ is dense}\} \enspace .
\]
Refer to \figref{dense-sparse}.
First we show that $F^+\supseteq F$ is not overly large:

\begin{figure}
  \begin{center}
	  \includegraphics[width=.98\textwidth]{figs/binary-tree-2}
  \end{center}
  \caption{An example with $\epsilon=1/4$.  Elements in $F$ are denoted by $\times$, dense nodes are shaded blue, sparse nodes whose right buddy is dense are shaded in pink. $F^+$ includes every value in every shaded node.}
\end{figure}

\begin{clm}\clmlabel{size}
  $F^+\supseteq F$ and, for any $\epsilon < 1/2$, $|F^+|\le (2+4\epsilon)|F|$.
\end{clm}

\begin{proof}
  Consider the set $M$ of \emph{maximal} dense nodes containing exactly
  those nodes $u$ of $T$ that are dense and such that no strict ancestor
  of $u$ is dense.  Observe that every $x\in F$ appears in exactly one
  node of $M$.  This immediately implies that $F^+\supseteq F$.

  To upper bound $|F^+|$, observe that $|F^+| \le \sum_{u\in M} 2|u|$
  and that 
  \[ 
     |F| = \sum_{u\in M} |F\cap u| \ge \sum_{u\in M} (1-\epsilon)|u| \enspace , 
  \]
	so
  \[
    \frac{|F^+|}{|F|} 
      \le \frac{2}{1-\epsilon} 
	= 2\sum_{r=0}^\infty\epsilon^r 
	= 2(1+\epsilon + \epsilon\sum_{r=1}^\infty \epsilon^r 
	\le 2(1+2\epsilon) \enspace . \qedhere
  \]
\end{proof}

\begin{clm}\clmlabel{path}
  For any $F\subset V(G)$, $G-F$ contains a monotone path 
  of size at least $n - (2+O(\epsilon))|F|$.
\end{clm}

\begin{proof}
  For a sparse node $u$ of $T$ with $x<\min(u)$ we say that a subset
  $u'\subset u$ is a \emph{witness set} for $(x,u)$ if $|u'|\ge
  (1-\epsilon/2)|u|-|F\cap u|$ and, for every $x'\in u$, $G-F$ contains
  a monotone path from $x$ to $x'$.

  We will construct the path in $G-F$ in a sequence of phases. At
  the beginning of phase $p$ we begin at a vertex $x_p\not\in F^+$
  and the phase will end when we are able to construct a monotone path to some
  vertex $y_p>x_p$, $y_p\not\in F^+$.  For the first phase, we begin with $x_1$
  as the first vertex not in $F^+$, i.e., $x_1=\min(V(G)\setminus F^+)$.
  For each subsequent phase $p$, the starting vertex $x_p=y_{p-1}$ found
  in the previous phase.

  The $p$th phase consists of a sequence of rounds. At the beginning of
  the $i$th round we have a triple $(a_i,u_i,u_i')$ where $a_i$ is a
  node of $T$ containing $x_p$, $u_i=r(a_i)$, and $u_i'$ is a witness
  set for $(x_p,u_i)$.  To begin phase $p$ we set $a_0=\{x_p\}$,
  $u_0=r(a_0)=\{x_p+1\}$, and $u_0'=u_0$.  Since $x_p\not\in F^+$,
  neither $a_0$ nor $u_0$ is dense.  Since $a_0$ and $u_0$ are buddies,
  $G$ contains the expander $B_{a_0u_0}$ which contains the single
  edge with endpoints $x_p$ and $x_p+1$.  Therefore, $u_0'=\{x_p+1\}$
  is indeed a witness set for $(x_p,u_0)$.

  Now consider a general round that begins with $(a_i,u_i,u_i')$.
  Since $x_p\in a_i$ and $x_p\not\in F^+$, each of $a_i$ and $u_i$ are
  sparse with respect to $F$. There are three cases to consider. These
  cases are not mutually exclusive, but when more than one applies,
  we consider the first applicable case.
  \begin{enumerate}
   \item $u_i'\setminus F^+$ is non-empty. In this case we take
    $y_p=\min(u_i'\setminus F^+)$. We have found a monotone
    path from $x_p$ to $y_p\not\in F^+$ so we complete the phase and
    being the next phase starting from $x_{p+1}=y_p$.

    \item $u_i$ contains $n-1$.  In this case, $r(u_i)$ does not exist.
    This is the final phase.  We take $y_p$ to be any element of $u_i'$.
    Since $y_p\in F^+$ we can not begin another phase and we terminate
    the entire procedure, having constructed a path that terminates at
    any node $y_p$.

    \item $a_i$ and $u_i$ are siblings with a common
    parent $a_{i+1}$.  In this case, we set $u_{i+1}=r(a_{i+1})$.
    Notice that $u_i'\subseteq u_{i}\subset a_{i+1}$ and that
    $|u_i'|\ge \epsilon|u_i|/2 = \epsilon|a_{i+1}|/4$.  Therefore, by
    \obsref{extend}, $u_{i+1}'= N_{B_{a_{i+1}u_{i+1}}}(u')\setminus F$
    is a witness set for $(x_p,u_{i+1})$.

   \item Otherwise, the parent $a_{i+1}$ of $a_i$ and the parent
   $u_{i+1}$ of $u_i$ are distinct (and $u_i$ is the left child
   of $u_{i+1}$.  Let $w_i=r(u_i)$ be the right child of $u_{i+1}$.
   Then $G$ contains the expander $B_{u_iw_i}$, so by \obsref{extend}
   $w_i'=N_{B_{uw}}(u')\setminus F$ is a witness set for $(x_p,w)$.

   Now, for every $y\in u_{i+1}'=u_i'\cup w_i'$, $G-F$ contains a monotone
   path from $x$ to $y$ and
   \begin{align*}
	   |u_{i+1}'| 
	   & \ge |u_i'|+|w_i'| \\
	   & \ge (1-\epsilon/2)|u_{i}| - |F\cap u_i| + (1-\epsilon/2)|w_i| - |F\cap w_i| \\
	   &= (1-\epsilon/2)|u_{i+1}| - |F\cap u_{i+1}|  \enspace .
   \end{align*}
  Therefore $u_{i+1}'$ is a witness set for $(x_p,u_{i+1})$.
  \end{enumerate}

  This completes the description of phase $p$ that constructs a
  monotone path in $G-P$ that begins at $x_p$
  and ends at $y_p$.  We will now argue that the vast majority of
  vertices in $\{x_p+1,\ldots,y_p-1\}$ are in the set $F^+$.  More
  specifically, we will show that $|F^+\cap\{x_p+1,\ldots,y_p-1\}| \ge
  (1-O(\epsilon))(y_p-x_p-1)$.

  Consider the sequence $u_0,\ldots,u_k$.  This sequence begins at
  $u_0=\{x_p+1\}$ and each set in the sequence is twice as large as its
  predecessor.   If, for some node $u_i$ in this sequence $u_i\supset
  u_{i-1}$, then $u_{i-1}$ is the left child of $u_i$ in $T$ and
  $w_{i-1}$ is the right child of $u_i$.    We create a new sequence
  $v_0,\ldots,v_k$ where
  \[
    v_i = \begin{cases} 
	    w_{i-1} & \text{if $i>0$ and $u_{i-1}\subset u_i$} \\
	    u_i & \text{otherwise}
    \end{cases}
  \]
  The nodes in the modified sequence $v_0,\ldots,v_k$ form a partition
  of $\{\min(v_0),\ldots,\max(v_k)\}$ and each node $v_i$ is at most
  four times the size of $v_{i-1}$.

  In Case~3, above, the phase will terminate in round $i+1$ if
  $|u_{i+1}\setminus F^+|>\epsilon|u_{i+1}|/2$.  In Case~4, the phase
  will terminate in round $i+1$ if $|w_i\setminus F^+|>\epsilon|w_i|/2$.
  This implies that, for each $i\in\{1,\ldots,k-1\}$, $|v_i\setminus
  F^+|\le\epsilon|v_i|/2$, otherwise the phase would have terminated
  in round $i<k$.  Stated another way,
  \[
	  |v_i|\le \frac{|v_i\cap F^+|}{1-\epsilon/2} \le (1+\epsilon)|v_i\cap F^+| \enspace .
  \]

  Let $\hat{v}_k=\{z\in v_k:z< y\}$.  Observe that $\hat{v}_k$ contains
  at most $\epsilon|v_k|/2$ vertices that are not in $F^+$, otherwise
  $v_k'\cap \hat{v}_k$ is non-empty, contradicting the minimality of $y_p$. 
  This means
  \[    |\hat{v}_k| \le |F^+\cap \hat{v}_k| + \epsilon|v_k|/2 \enspace .
  \]
  Now we have
  \begin{align*}
	  y_p - x_p - 1 & = \sum_{i=0}^{k-1} |v_i| + |\hat{v}_k|  \\
	  & \le (1+\epsilon)\sum_{i=0}^{k-1} |v_i\cap F^+| + |\hat{v}_k| \\
	  & \le (1+\epsilon)\sum_{i=0}^{k-1} |v_i\cap F^+| + |\hat{v}_k\cap F^+| + \epsilon|v_k|/2 \\ 
	  & \le  (1+\epsilon)\sum_{i=0}^{k-1}|v_i\cap F^+| + |\hat{v}_k\cap F^+| + 2\epsilon|v_{k-1}| \\ 
	  & \le  (1+\epsilon)\sum_{i=0}^{k-1}|v_i\cap F^+| + |\hat{v}_k\cap F^+| + (2\epsilon+2\epsilon^2)|v_{k-1}\cap F^+| \\ 
	  & \le  (1+3\epsilon+\epsilon^2)\sum_{i=0}^{k-1}|v_i\cap F^+| + |\hat{v}_k\cap F^+| \\
	  & <  (1+4\epsilon)\sum_{i=0}^{k-1}|v_i\cap F^+| + |\hat{v}_k\cap F^+|  \\
	  & =  (1+4\epsilon)|\{z\in F^+: x_p< z< y_p\}|
  \end{align*}

  Suppose the procedure terminates after $s-1$ phases, so that $x_1<\cdots<x_{s-1}$ are the vertices that begin each phase.  For convenience we define $x_s=y_{s-1}$, $x_0=-1$ and $x_{s+1}=n$.

  For each $p\in\{0,\ldots,s\}$, let $F_p^+=\{z\in F^+: x_p <
  x < x_{p+1}\}$.  Since none of $x_1,\ldots,x_s$ are in $F^+$,
  $F_0^+,\ldots,F_{s}$ is a partition of $F^+$.  

 For each $p\in\{1,\ldots,s\}$, the argument above shows that $(1+4\epsilon)|F_i^+|\ge x_{p+1}-x_p-1$.
  Furthermore, the minimality of $x_1$ ensures that $(1+4\epsilon)|F_0^+|\ge |F_0^+|=x_1=x_1-x_0-1$.  
   Therefore,
  \begin{align*}
	  (1+4\epsilon)|F^+| & = (1+4\epsilon)\sum_{i=0}^s |F^+_i| \\
		& \ge \sum_{i=0}^s (x_{i+1}-x_i-1) \\
		& = n-s \enspace .
  \end{align*}
  so 
  \begin{align*}
	  s & \ge n-(1+4\epsilon)|F^+| \\
	  & \ge n-(1+4\epsilon)(2+4\epsilon)|F|\\
	  & =n-(2+O(\epsilon))|F| \enspace . 
  \end{align*}
  Since $s$ is a lower bound on the size of the monotone path
  constructed by this procedure, this completes the proof.
\end{proof}

\begin{cor}
	The graph $G$ is a $(2+O(\epsilon))k$-robust 1-spanner.
\end{cor}

\begin{proof}
	For any $F\subset V(G)$, \clmref{path} constructs a monotone path $P$ in $G-F$ of size $n-(2+O(\epsilon))|F|$.  Let $F^*=V(G)\setminus V(P)$ so $|F^*|\le (2+O(\epsilon))|F|$.  Then $P$, which is a subgraph of $G-F$ is a 1-spanner of $V(G)\setminus F^*$. Therefore $G-F$ is a 1-spanner of $V(G)\setminus F^*$, as required.
\end{proof}
%  
%  
%  For each such node, we replace $u_i$ with $w_i=r_{i-1}$
%
%	
%	
%	Now, modify this sequence so that, for each $i\in\{2,\ldots,k\}$, if $u_i$ contains $u_{i-1}$ if a node contains its predecessor
%	
%	
%	$u_0,\ldots,u_k$ defined as follows: $u_0=\{x+1\}$ and, for each 
%  $i\in\{1,\ldots,k-1\}$
%  \begin{enumerate}	
%    \item if $u_i$ is the left child of its parent, then $u_i$ is the parent of $u_{i-1}$, if $u_{i-1}$ is 
%		  
%		  
%		  
%		  $u_i$In Case~1, $u_{i}$ is the parent of $r(u_{i-1})$.
%     \item In Case~2, $u_i$ is the parent of (its left child) $u_{u-1}$.
%  \end{enumerate}
%  For each $i\in\{1,\ldots,k-1\}$,
%  \[
%	  |u_i\setminus F^+| \le (\epsilon/2)|u_i| \le \epsilon|F^+\cap u_i|
%  \]
%  where the second inquality holds provided that $\epsilon \le 2/3$.
%  For $u_k$ TODO: be more careful to partition here.
%	
%	
%	$\hat{u}\not\subseteq F^+$
%  and it is time to complete the phase.  
%	
%  We claim that, at this point, $w=r(u)$ is dense.  Indeed, since
%  $u\subseteq F^+$ and $u$ is sparse, the only other possibility is
%  that $u$ has some strict ancestor $\hat\hat{u}$ that is dense or has
%  a right buddy that is dense.  In Case~1 above, this is not possible
%  because every strict ancestor $\hat\hat{u}$ of $u$ contains $x$ but we
%  know that $x\not\in F^+$.  In Case~2 above, this is also not possible
%  because every strict ancestor $\hat\hat{u}$ of $u$ is an ancestor of
%  $\hat{u}$ and we know that $\hat{u}\not\subseteq F^+$.
%
%  Now, in Case~2 above, if $w$ is dense, then $w\subseteq F^+$ and
%  therefore $u\subseteq F^+$, so $\hat{u}=u\cup w\subseteq F^+$.
%  Therefore, a phase can never complete after Case~2.  Therefore,
%  the phase completes in Case~1 so the left child $w$ of $\hat{u}$
%  is dense and, since $\hat{u}$ is not dense the right child
%  $r(w)$ of $\hat{u}$ must be sparse.  By the same reasoning, the right child of $
%  
%  
%  , so $|r(w)\setminus F|>
%  \epsilon|r(w)|=\epsilon|\hat{u}|/2$.
%
%
%  To do this we take the first vertex $y=\min(\hat{u}')$ in $\hat{u}$
%  that can be reached by a monotone path from $x$.
%
%
%  Note that the monotone path from $x$ does not contain any of the
%  elements $\{z\in \hat{u}: z < y\}$.  None of these elements are in $F^+$
%  so we must take care to prove that the monotone path is, indeed, long.
%  Since we are assuming that $\hat{u}\not\subseteq F^+$, this means that
%  no ancestor of $\hat{u}$ is dense or has a right buddy that is dense.
%  On the other hand, $u\subseteq F^+$ since, otherwise, the phase would
%  have completed upon reaching $u$.  
%  
%
%  Therefore, $w=r(u)$ is dense.
%
%
%
%  Suppose that, after the completion of a round, 
%  $u\not\subset F^+$, 
%	
%	
%	then we have a monotone path from $x$ 
%	
%	satisifies our requirement
%
%
%
%
%  We being the phase at the leaf $u$ of $T$ that contains only $x$.
%  Since $x\not\in F^+$, we know that, for each ancestor $w$ of $u$, $w$
%  and $r(w)$ are both sparse.  The preceding statement is true even for
%  $w=u$, so we know that $r(u)$ is sparse.  The pair $(u,w)$ are a buddy
%  pair so $G$ contains the expander $B_{uw}$ which consists of a single
%  edge joining $x$ and $x+1$. If $w\not\subseteq F^+$, then we add the
%  edge $(x,x+1)$ to our path and the phase is complete.
%
%    Otherwise, we set $u=w$ and continue from $u$.  From this point onward,
%  $u\subset F^+$ will be a sparse node that contains a subset $u'\subset
%  u$, $|u'|\ge \epsilon|u|/2$, such that, for each $x'\in u'$, $G-F$
%  contains a monotone path from $x$ to $x'$.  
%
%  The fact that $u\subset F^+$ implies that $u$ has an ancestor $w$ such that $w$ or $r(w)$ is dense.  (This includes the case where $w=u$ and $r(u)$ is dense.)
%	
%	
%	Note that, in this case $G-F$ contains a monotone path from $x$ to every vertex in $u\setminus F$.
%
%	
%	From this we conclude that $r(u)$ is sparse.
%
%  During a phase we will have a specific node $u$ of $T$ and a subset
%	
%	for which we know thereare 
%
%
%
%  Let $v_0,\ldots,v_{\log_2 n}$ be the path from the leaf of $T$
%  containing $x$ to the root of $T$ and let $v_t$ be the last node on
%  this path that is the left child of its parent.  Observe that, for
%  each $j\in\{0,\ldots,t\}$, $v_j$ has a right neighbour $v_j'$ and,
%  since $x\not\in F^*$, $v_j'$ is sparse.
%
%  We claim that, for each $j\in\{0,\ldots,k\}$ and each node $y\in
%  v_j'\setminus F$, $G-F$ contains a monotone path from $x$ to $y$.
%  We prove this claim by induction on $j$.  For the base case, $v_0=\{x\}$
%  and $v_0'$ are buddy leaves in $T$ and there is an expander graph
%  for $(v_0,v_0')$
%  that contains the edge $xy$.
%
%  For $j>1$, the inductive hypothesis states that $G-F$ contains
%  a monotone path from $x$ to each $x'\in v_{j-1}'\setminus F$.
%  Let $S=v_{j-1}'\setminus F$.  There are two cases to consider:
%  \begin{enumerate}
%    \item $v_{j-1}$ is the left child $v_{j}$.  In this case,
%	  $S$ is a subset of $v_j$ of size at least $\epsilon|v_{j-1}|
%	  = \epsilon|v_j|/2$.  The graph $G$ contains an expander graph
%	  for the buddy pair $(v_j,v_j')$ and therefore every $y\in v_j'$
%	  contains a neighbour $x'\in S$.  The (inductive) path from
%	  $x$ to $x'$ plus the edge $x'y$ is a monotone path in $G-F$
%	  from $x$ to $y$, as required.
%
%    \item $v_{j-1}$ is the right child of $v_j$.	In this case, let
%	  $v_{j-1}''$ be the right neighbour of $v_{j-1}'$.
%	  The graph $G$ contains an expander graph for the pair
%	  $(v_{j-1}',v_{j-1}'')$ and $S$ is a subset of $v_{j-1}'$ of
%	  size at least $\epsilon|v_{j-1}'|$ so every $y\in v_{j-1}''$ is
%	  adjacent to some $x'\in v_{j-1}'$.  Therefore, for every $y\in
%	  v_{j-1}''$, there is a monotone path from $x$ to $x'$ to $y$.
%	  We finish by observing that $v_j' = v_{j-1}'\cup v_{j-1}''$.
%  \end{enumerate}
%  Finally, we observe that, for any $y\ge x$, $y$ is contained in at least one of of $v_0',\ldots,v_t'$.  Therefore $G-F$ contains a monotone path from $x$ to each $y\in\{x+1,\ldots,n\}\setminus F$, as required.
%\end{proof}
%
%\begin{prop}
%  The graph $G$ defined above has $O(\epsilon^{-1}n\log n)$ edges
%  and, for any $F\subset V(G)$, $G-F$ contains a monotone path of
%  size at least $n-(2+4\epsilon)|F|$.
%\end{prop}
%
%\begin{proof}
%	Let $x_1<x_2<\cdots<x_m$ be the elements of $V(G)\setminus F^*$.
%	Then, by \clmref{path}, there is a monotone path from $x_i$
%	to $x_{i+1}$, for each $i\in\{1,\ldots,m-1\}$.
%	The concatentation of all these paths is a monotone path of size $n-|F^*| \ge n-(2+4\epsilon)|F|$, by \clmref{size}.
%\end{proof}
%
%\section{Achieving $1+\epsilon$}
%
%
%Next, we show how to generalize our construction to obtain a set $F^*$
%of size at most $(1+\epsilon)|F|$.  To achieve this, we add additional
%expander graphs. For each $u$ of $T$, let $r(u)$ denote $u$'s right buddy.
%Our graph $G$ is parameterized by a natural number $q$.  For each
%node $u$ of $T$, and each $i\in\{1,\ldots,q\}$, $G$ contains the edges
%of an expander graph for the pair $(u,r^{(i)}(u))$.  Observe that
%this construction, with $q=1$, corresponds exactly to the graph $G$
%described in the previous section.  In this generalized construction,
%the number of edges is $O(q\epsilon^{-1} n\log n)$ edges.
%
%Given a set $F\subset V$, we define $F^*$ as follows: Each node $u$ of $T$
%is included in $F^*$ if
%\begin{enumerate}
%   \item $u$ is dense with respect to $F$; or
%   \item $r^{(i)}(u)$ exists and is dense with respect to 
%	 $F$ for all $i\in\{1,\ldots,q\}$.
%\end{enumerate}
%It follows that 
%\begin{align*} 
%	|F^*| & \le \left(\frac{1+1/q}{1-\epsilon}\right)|F| \\
%	& \le (1+1/q)(1+2\epsilon)|F| &\text{(for any $\epsilon \le 1/2$)} \\
%	& \le (1+4\epsilon)|F| &\text{(for $q=\lceil 1/\epsilon\rceil$).}
%\end{align*}
%
%All that remains is to prove that $G-F$ contains a monotone path that includes every vertex in $V\setminus F^*$.
%
%\begin{clm}
%  For any $x\in V\setminus F^*$ and any $y\in \{x+1,\ldots,n\}\setminus
%  F$, $G-F$ contains a monotone path from $x$ to $y$.
%\end{clm}
%
%\begin{proof}
%  Let $v_0,\ldots,v_{\log_2 n}$ be the path from the leaf of $T$
%  containing $x$ to the root of $T$.  Let $v_t$ be the last node on
%  this path such that at least one of $r^{(1)}(v_t),\ldots,r^{(q)}(v_t)$
%  is defined and is sparse.  Observe that every vertex $y > \max(v_t)$
%  is contained in $F^*$.
%
%  We claim that for every $j\in\{1,\ldots,t\}$ and every node $w$ such
%  that $w=r^{(i)}(v_j)$, for some $i\in\{1,\ldots,q\}$, $G-F$ contains a
%  monotone path from $x$ to every vertex $y\in w\setminus F$.  We prove
%  this by claim by induction on $j$.  When $j=0$, $w = r^{(i)}(v_0)$
%  for some $i\in\{1,\ldots,q\}$, so $G$ contains an expander graph for
%  the pair $(v_0,w)$ and this graph contains the single edge joining $x$
%  to the unique vertex in $y\in w$.
%
%  Now we wish to prove this claim for any $w\in\{r^{(i)}(v_j): i\in\{1,\ldots,q\}\}$.  First observe that, for any $i\in\{1,\ldots,\lfloor(q-1)/2\rfloor\}$, the two children of $r^{(i)}(v_j)$ are contained in 
%	$\{r^{(i)}(v_{j-1}): i\in\{1,\ldots,q\}\}$. Therefore, by induction on $j$ there is a monotone path from $x$ to every vertex of $r^{(i)}(v_j)$ for every $i\in\{1,\ldots,\lfloor (q-1)/2\rfloor\}$. From this point on we therefore assume that $w=r^{(i)}(v_j)$ for some $i \ge $ TODO: Continue here!
%	
%	node $w$
%
%
%  For each $j\in\{1,\ldots,t\}$, let $v_j'$ denote the first node among
%  $r^{(1)}(v_j),\ldots,r^{q}(v_j)$ that is sparse.  To prove our claim
%  for $j>1$, consider the node $v_{j-1}'$.  From the definition of $F^*$
%  (and $t$) the node $v_{j-1}'$ exists.  Let $u$ be the parent
%  of $v_{j-1}'$.
%  Below, we will argue
%  that $w=u$ or that $w=r^{(i)}(u)$ for some $i\in\{1,\ldots,q\}$.
%  \begin{itemize}
%    \item In the case that $w=u$, the node $u$ has two children $v_{j-1}'$
%    and $v$.  If $v_{j-1}'$ is the right child of $u$ then, by induction,
%    there is a monotone path from $x$ to every vertex in $v\setminus F$
%    and a monotone path from $x$ to every vertex in $v_{j-1}'\setminus F$
%    and we are done.  If $v_{j-1}'$ is the left child of $u$, then there
%    is a monotone path from $x$ to every vertex in $x'\in v_{j-1}'$.
%    Now, $v_{j-1}'\cap F$ is a subset of $v_{j-1}'$ of size at least
%    $\epsilon |v_{j-1}'|$ and $G$ contains an expander graph for the
%    pair $(v_{j-1}',v)$.	Therefore, for every $y\in v$ there is
%    vertex in $x'\in v_{j-1}'$ adjacent to $y$ and therefore there is
%    a monotone path from $x$ to $x'$ to $y$, as required.
%
%    \item In the case that $w=r^{(i)}(u)$ for some $i\in\{1,\ldots,q\}$,
%    $v_{j-1}'\setminus F$ is a subset of $u$ of size at least
%    $\epsilon|v_{j-1}'| \ge \epsilon|u|/2$.  Now $G$ contains an expander
%    graph for the pair $(u,w)$.  Therefore every vertex $y\in w$ is
%    adjacent to some vertex $x'\in v_{j-1}'\setminus F$.  By induction there
%    is a monotone path from $x$ to $x'$ and the edge $x'y$ extends this
%    to a monotone path from $x$ to $y$, as required.
%  \end{itemize}
%
%  Thus, all that remains is to show that $w=u$ or $w=r^{(i)}(u)$ for
%  some $i\in\{1,\ldots,q\}$.  Let $\tau\in\{0,\ldots,q-1\}$ be such that
%  $v_{j-1}'=r^{(\tau+1)}(v_j)$.  From the definition of $v_{j-1}'$, we
%  know that $r^{(1)}(v_{j-1}),\ldots,r^{(\tau)}(v_{j-1})$ are all dense
%  nodes.  Any node of $T$ with 2 dense children is itself dense. Therefore
%  $r^{(1)}(v_j),\ldots,r^{(\lfloor\tau/2\rfloor)}(v_j)$ are all dense.
%  Therefore, $w=r^{(\kappa)}(v_j)$ for some $\kappa >
%  \lfloor\tau/2\rfloor$.  On the other hand, $u=r^{(\gamma)}(v_j)$ for
%  some $\gamma\in\{\lfloor\tau/2\rfloor,\lfloor\tau/2\rfloor+1\}$.
%  In either case $\kappa \ge \gamma$ so $w=u$ or
%  $w=r^{(\kappa-\gamma)}(u)$, as required.
%\end{proof}
%
%\section{Crap}
%
%Then $\|vw\|_{G-F}=\infty$ so $G-F$ is a not a $t$-spanner of $V\setminus F=V\setminus F^*$ for any $t<\infty$.}
%A slightly more sophisticated argument shows that any $(1+1/d)k$-robust $t$-spanner must have $\Omega(dn)$ edges.\footnote{Proof: If $G$ has $
%
%rules out the existence
%of $(k+o(k))$-robust spanners having a subquadratic number of edges.
%
%
%

\bibliographystyle{plain}
\bibliography{robust2}

\end{document}









