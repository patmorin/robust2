\documentclass{patmorin}
\usepackage{pat}
\usepackage{hyperref}
\usepackage[noend]{algorithmic}
\hypersetup{colorlinks=true, linkcolor=linkblue,  anchorcolor=linkblue,
citecolor=linkblue, filecolor=linkblue, menucolor=linkblue,
urlcolor=linkblue, pdfcreator=Me, pdfproducer=Me} 
\setlength{\parskip}{1ex}

\title{\MakeUppercase{Optimal Robust Spanners in One Dimension}}

\author{Prosenjit Bose, Paz Carmi, Michael van Dyk, Pat Morin, 
  \newline
  Luis Fernando Schulz Xavier da Silveira,
  and Jan Volec}

\newcommand{\note}[2]{{\color{red}[#1:~#2]}}

\begin{document}
\maketitle


\begin{abstract}
  For any $\epsilon >0$ and any $n\in\N$, we construct a graph $G=(V,E)$
  with vertex set $V=\{0,\ldots,n-1\}$ having $O(\epsilon^{-2}n\log n)$
  edges, and such that, for any $F\subseteq V$, $G-F$ contains a monotone
  path of length at least $n-(1+\epsilon)|F|$.  The number of edges
  in $G$ matches a lower-bound of Bose \etal\ (2013) and improves the
  previous construction of Buchin, Hulshof and Ol\'ah (2018) which has
  $O(n^{1+\epsilon})$ edges and guarantees $|F^*|\le c^{1/\epsilon}|F|$,
  for some constant $c\ge 3$ and any $\epsilon >0$.
\end{abstract}

\section{Introduction}

A geometric graph $G=(V,E)$ with vertex set $V\subset\R^d$ is a (geometric)
$t$-spanner of a subset $X\subset V$ if, for every pair of vertices
$v,w\in X$, $v\neq w$,
\begin{equation}
	\frac{\|vw\|_G}{\|vw\|} \enspace , \eqlabel{spanning-ratio}
\end{equation}
where $\|vw\|$ denotes the Euclidean distance between $v$ and $w$ and
$\|vw\|_G$ denote the Euclidean length of the shortest path between $v$
and $w$ in $G$, where use the convention that $\|vw\|_G=\infty$ if $v$
and $w$ are in different components of $G$.  
Most of the research on spanners focus on \emph{sparse}
spanners, where the number of edges in $G$ is linear, or close to linear,
in $|V|$.  In addition to having natural applications to transportation
networks, sparse $t$-spanners have found numerous applications in
approximation algorithms and geometric data structures.

For any non-decreasing function $f\colon\N\to\N$, Bose \etal\ say that
a geometric graph $G$ is an \emph{$f(k)$-robust $t$-spanner} if, for
every set $F\subset V(G)$, there exists a set $F^*\supseteq F$ such that
$|F^*|\le f(|F|)$ and the graph $G-F$ is a $t$-spanner of $V(G)\setminus
F^*$.  In networking applications, this definition captures the idea
that a set $F$ of faulty vertices should only have a noticeable effect
on a set $F^*$ of vertices whose size is a function of the size of $F$.

Under this definition, the most robust spanner one could hope for
would be a $k$-robust spanner, but it is straightforward to argue
that, even for one dimensional point sets, the complete graph is the
only $k$-robust spanner.\footnote{Proof: Consider any pair of vertices
$v,w\in V$ that are not adjacent in $G$ and let $F=V\setminus\{v,w\}$.
Then $\|vw\|_{G-F}=\infty$ so $G-F$ is a not a $t$-spanner of $V\setminus
F=V\setminus F^*$ for any $t<\infty$.} The natural next-best choice is a
$ck$-robust spanner with a near-linear number of edges, for some $c>1$ .

Bose \etal \cite[Theorem~3]{bose.dujmovic.ea:robust} show that for any
constants $c,t>1$ and any $n$-element set $V\subset\R$, any $ck$-robust
$t$-spanner $G=(V,E)$ must have $\Omega(n\log n)$ edges.  Their work
does not give any construction matching this lower bound, leaving a
gap in our understanding of robust spanners, even in one dimension.
Buchin, Hulshof and Ol\'ah \cite{buchin.hulshof.olah:robust} made the
first progress in filling this gap by constructing, for any $\delta>0$,
a $k2^{O(1/\delta)}$-robust spanner 1-spanner having $O(n^{1+\delta})$
edges.  In the current paper, we prove the following result:

\begin{thm}\thmlabel{main}
  For every $n$-element set $V\subset\R$ and every $\epsilon>0$,
  there exist a $(1+\epsilon)k$-robust 1-spanner $G=(V,E)$ with
  $O(\epsilon^{-2}n\log n)$ edges.
\end{thm}

The lower-bound discussed above implies that the number of edges
in \thmref{main} is optimal, except possibly for the dependence on
$\epsilon$.


\section{The Construction}


Without loss of generality, assume $n$ is a power of 2.  Construct a
complete ordered binary tree $T$ of height $\log_2 n$ whose nodes are
contiguous sets of integers..  The leaves of $T$, in left-to-right order,
are the singleton sets $\{0\},\ldots,\{n-1\}$ and each internal node of
$T$ is the set obtained by taking the union of its two children.  Stated
another way, $T=T(0,\log_2 n-1)$, whree $T(i,b)$ has a single node $\{i\}$
if $b=0$. Otherwise $T(i,b)$ has a root node $\{i,i+1,\ldots,i+2^{b-1}-1\}$
whose two children are the subtrees $T(i,b-1)$ and $T(i+2^{b-1}, b-1)$.

Note that the nodes of $T$ can be partitioned in levels $L_0,\ldots,L_{\log_2 n}$, where each $L_i=\{u_{i,0},\ldots,u_{i,2^i-1}\}$ and
\[
	u_{i,k} = \{kn/2^i,\ldots,(k+1)n/2^{i}-1\} \enspace .
\]
For each consecutive pair $(u_{i,k},u_{i,k+1})$ of nodes in $L_i$ we say
that $u_{i,k}$ and $u_{i,k+1}$ are \emph{buddies} and we call $u_{i,k+1}$
the \emph{right buddy} of $u_{i,k}$.

\subsection{A First Try}

We now describe a first construction of $G$ that has $O(\epsilon^{-1}n\log
n)$ edges and guarantees that, for any $F\subset V$,  $G-F$ has a monotone
path of length at least $n-(2+4\epsilon)|F|$.

The graph $G$ contains, for each $i\in\{1,\ldots,\log_2 n\}$ and each
$k\in\{0,\ldots,2^{i}-2\}$ the edges of a bipartite \emph{expander graph}
$B_{i,k}$.  The vertex set of $B_{i,k}$ has two classes $A=u_{i,k}$ and
$B=u_{i,k+1}$ and has the following property:  For any subset $S\subset
A$ with $|S|\ge \epsilon |A|$, and every vertex $y$ in $B$, there is
some $x\in A$ such that $xy$ is an edge of $B_{i,k}$. It is known that
such expanders exist in which every node has degree $O(\epsilon^{-1})$
\cite{X}.  Observe that for every pair of buddies $(u,u')$ in $T$,
there is an expander graph between $u$ and $u'$.

Each node $u_{i,k}$ of $T$ is involved in at most two expander
graphs $B_{i,k-1}$ and $B_{i,k}$ and each such graph contributes
$O(\epsilon^{-1}n/2^{i})$ edges to $G$.  Therefore, each level $L_i$
contributes at most $O(\epsilon^{-1} n)$ edges to $G$ for a total of
$O(\epsilon^{-1}n\log n)$ edges.  

Next we define our set $F^*\supseteq F$.  Given a set $F\subseteq V$,
we say that a node $u$ of $T$ is \emph{sparse} (with respect to $F$)
if $|u\cap F| \ge (1-\epsilon)|u|$ and $u$ is \emph{dense} (with respect
to $F$) otherwise.  We then define
\[
	F^* = \cup\{ u_{i,k}\in V(T): \text{$u_{i,k}$ is dense or $u_{i,k+1}$ is dense} \} \enspace .
\]
That is, a node $u$ of $T$ contributes to $F^*$ if $u$ is dense or $u$'s
right buddy is dense.

First we show that $F^*\supseteq F$ is not overly large:

\begin{clm}\clmlabel{size}
  $F^*\supseteq F$ and, for any $\epsilon < 1/2$, $|F^*|\le (2+4\epsilon)|F|$.
\end{clm}

\begin{proof}
  Consider the set $M$ of \emph{maximal} dense nodes containing exactly
  those nodes $u$ of $T$ that are dense and such that no ancestor of $u$
  is dense.  Observe that every $x\in F$ appears in exactly one node
  of $M$.  This immediately implies that $F^*\supseteq F$.

  To upper bound $|F^*|$, observe that $|F^*| \le \sum_{u\in M} 2|u|$
  and that 
	$|F| \ge \sum_{u\in M} (1-\epsilon)|u|$, 
	so
  \[
    \frac{|F^*|}{|F|} 
      \le \frac{2}{1-\epsilon} 
	= 2\sum_{r=0}^\infty\epsilon^r 
	= 2(1+\epsilon + \epsilon\sum_{r=1}^\infty \epsilon^r 
	\le 2(1+2\epsilon) \enspace . \qedhere
	  \]
\end{proof}

Next we show that $G-F$ contains a monotone path of length $n-|F^*|$.

\begin{clm}\clmlabel{path}
  For any $x\in V\setminus F^*$ and any $y\in \{x+1,\ldots,n\}\setminus
  F$, $G-F$ contains a monotone path from $x$ to $y$.
\end{clm}

\begin{proof}
  Let $v_0,\ldots,v_{\log_2 n}$ be the path from the leaf of $T$
  containing $x$ to the root of $T$ and let $v_t$ be the last node on
  this path that is the left child of its parent.  Observe that, for
  each $j\in\{0,\ldots,t\}$, $v_j$ has a right neighbour $v_j'$ and,
  since $x\not\in F^*$, $v_j'$ is sparse.

  We claim that, for each $j\in\{0,\ldots,k\}$ and each node $y\in
  v_j'\setminus F$, $G-F$ contains a monotone path from $x$ to $y$.
  We prove this claim by induction on $j$.  For the base case, $v_0=\{x\}$
  and $v_0'$ are buddy leaves in $T$ and there is an expander graph
  for $(v_0,v_0')$
  that contains the edge $xy$.

  For $j>1$, the inductive hypothesis states that $G-F$ contains
  a monotone path from $x$ to each $x'\in v_{j-1}'\setminus F$.
  Let $S=v_{j-1}'\setminus F$.  There are two cases to consider:
  \begin{enumerate}
    \item $v_{j-1}$ is the left child $v_{j}$.  In this case,
	  $S$ is a subset of $v_j$ of size at least $\epsilon|v_{j-1}|
	  = \epsilon|v_j|/2$.  The graph $G$ contains an expander graph
	  for the buddy pair $(v_j,v_j')$ and therefore every $y\in v_j'$
	  contains a neighbour $x'\in S$.  The (inductive) path from
	  $x$ to $x'$ plus the edge $x'y$ is a monotone path in $G-F$
	  from $x$ to $y$, as required.

    \item $v_{j-1}$ is the right child of $v_j$.	In this case, let
	  $v_{j-1}''$ be the right neighbour of $v_{j-1}'$.
	  The graph $G$ contains an expander graph for the pair
	  $(v_{j-1}',v_{j-1}'')$ and $S$ is a subset of $v_{j-1}'$ of
	  size at least $\epsilon|v_{j-1}'|$ so every $y\in v_{j-1}''$ is
	  adjacent to some $x'\in v_{j-1}'$.  Therefore, for every $y\in
	  v_{j-1}''$, there is a monotone path from $x$ to $x'$ to $y$.
	  We finish by observing that $v_j' = v_{j-1}'\cup v_{j-1}''$.
  \end{enumerate}
  Finally, we observe that, for any $y\ge x$, $y$ is contained in at least one of of $v_0',\ldots,v_t'$.  Therefore $G-F$ contains a monotone path from $x$ to each $y\in\{x+1,\ldots,n\}\setminus F$, as required.
\end{proof}

\begin{prop}
  The graph $G$ defined above has $O(\epsilon^{-1}n\log n)$ edges
  and, for any $F\subset V(G)$, $G-F$ contains a monotone path of
  size at least $n-(2+4\epsilon)|F|$.
\end{prop}

\begin{proof}
	Let $x_1<x_2<\cdots<x_m$ be the elements of $V(G)\setminus F^*$.
	Then, by \clmref{path}, there is a monotone path from $x_i$
	to $x_{i+1}$, for each $i\in\{1,\ldots,m-1\}$.
	The concatentation of all these paths is a monotone path of size $n-|F^*| \ge n-(2+4\epsilon)|F|$, by \clmref{size}.
\end{proof}

\section{Achieving $1+\epsilon$}


Next, we show how to generalize our construction to obtain a set $F^*$
of size at most $(1+\epsilon)|F|$.  To achieve this, we add additional
expander graphs. For each $u$ of $T$, let $r(u)$ denote $u$'s right buddy.
Our graph $G$ is parameterized by a natural number $q$.  For each
node $u$ of $T$, and each $i\in\{1,\ldots,q\}$, $G$ contains the edges
of an expander graph for the pair $(u,r^{(i)}(u))$.  Observe that
this construction, with $q=1$, corresponds exactly to the graph $G$
described in the previous section.  In this generalized construction,
the number of edges is $O(q\epsilon^{-1} n\log n)$ edges.

Given a set $F\subset V$, we define $F^*$ as follows: Each node $u$ of $T$
is included in $F^*$ if
\begin{enumerate}
   \item $u$ is dense with respect to $F$; or
   \item $r^{(i)}(u)$ exists and is dense with respect to 
	 $F$ for all $i\in\{1,\ldots,q\}$.
\end{enumerate}
It follows that 
\begin{align*} 
	|F^*| & \le \left(\frac{1+1/q}{1-\epsilon}\right)|F| \\
	& \le (1+1/q)(1+2\epsilon)|F| &\text{(for any $\epsilon \le 1/2$)} \\
	& \le (1+4\epsilon)|F| &\text{(for $q=\lceil 1/\epsilon\rceil$).}
\end{align*}

All that remains is to prove that $G-F$ contains a monotone path that includes every vertex in $V\setminus F^*$.

\begin{clm}
  For any $x\in V\setminus F^*$ and any $y\in \{x+1,\ldots,n\}\setminus
  F$, $G-F$ contains a monotone path from $x$ to $y$.
\end{clm}

\begin{proof}
  Let $v_0,\ldots,v_{\log_2 n}$ be the path from the leaf of $T$
  containing $x$ to the root of $T$.  Let $v_t$ be the last node on
  this path such that at least one of $r^{(1)}(v_t),\ldots,r^{(q)}(v_t)$
  is defined and is sparse.  Observe that every vertex $y > \max(v_t)$
  is contained in $F^*$.

  We claim that for every $j\in\{1,\ldots,t\}$ and every node $w$ such
  that $w=r^{(i)}(v_j)$, for some $i\in\{1,\ldots,q\}$, $G-F$ contains a
  monotone path from $x$ to every vertex $y\in w\setminus F$.  We prove
  this by claim by induction on $j$.  When $j=0$, $w = r^{(i)}(v_0)$
  for some $i\in\{1,\ldots,q\}$, so $G$ contains an expander graph for
  the pair $(v_0,w)$ and this graph contains the single edge joining $x$
  to the unique vertex in $y\in w$.

  Now we wish to prove this claim for any $w\in\{r^{(i)}(v_j): i\in\{1,\ldots,q\}\}$.  First observe that, for any $i\in\{1,\ldots,\lfloor(q-1)/2\rfloor\}$, the two children of $r^{(i)}(v_j)$ are contained in 
	$\{r^{(i)}(v_{j-1}): i\in\{1,\ldots,q\}\}$. Therefore, by induction on $j$ there is a monotone path from $x$ to every vertex of $r^{(i)}(v_j)$ for every $i\in\{1,\ldots,\lfloor (q-1)/2\rfloor\}$. From this point on we therefore assume that $w=r^{(i)}(v_j)$ for some $i \ge $ TODO: Continue here!
	
	node $w$


  For each $j\in\{1,\ldots,t\}$, let $v_j'$ denote the first node among
  $r^{(1)}(v_j),\ldots,r^{q}(v_j)$ that is sparse.  To prove our claim
  for $j>1$, consider the node $v_{j-1}'$.  From the definition of $F^*$
  (and $t$) the node $v_{j-1}'$ exists.  Let $u$ be the parent
  of $v_{j-1}'$.
  Below, we will argue
  that $w=u$ or that $w=r^{(i)}(u)$ for some $i\in\{1,\ldots,q\}$.
  \begin{itemize}
    \item In the case that $w=u$, the node $u$ has two children $v_{j-1}'$
    and $v$.  If $v_{j-1}'$ is the right child of $u$ then, by induction,
    there is a monotone path from $x$ to every vertex in $v\setminus F$
    and a monotone path from $x$ to every vertex in $v_{j-1}'\setminus F$
    and we are done.  If $v_{j-1}'$ is the left child of $u$, then there
    is a monotone path from $x$ to every vertex in $x'\in v_{j-1}'$.
    Now, $v_{j-1}'\cap F$ is a subset of $v_{j-1}'$ of size at least
    $\epsilon |v_{j-1}'|$ and $G$ contains an expander graph for the
    pair $(v_{j-1}',v)$.	Therefore, for every $y\in v$ there is
    vertex in $x'\in v_{j-1}'$ adjacent to $y$ and therefore there is
    a monotone path from $x$ to $x'$ to $y$, as required.

    \item In the case that $w=r^{(i)}(u)$ for some $i\in\{1,\ldots,q\}$,
    $v_{j-1}'\setminus F$ is a subset of $u$ of size at least
    $\epsilon|v_{j-1}'| \ge \epsilon|u|/2$.  Now $G$ contains an expander
    graph for the pair $(u,w)$.  Therefore every vertex $y\in w$ is
    adjacent to some vertex $x'\in v_{j-1}'\setminus F$.  By induction there
    is a monotone path from $x$ to $x'$ and the edge $x'y$ extends this
    to a monotone path from $x$ to $y$, as required.
  \end{itemize}

  Thus, all that remains is to show that $w=u$ or $w=r^{(i)}(u)$ for
  some $i\in\{1,\ldots,q\}$.  Let $\tau\in\{0,\ldots,q-1\}$ be such that
  $v_{j-1}'=r^{(\tau+1)}(v_j)$.  From the definition of $v_{j-1}'$, we
  know that $r^{(1)}(v_{j-1}),\ldots,r^{(\tau)}(v_{j-1})$ are all dense
  nodes.  Any node of $T$ with 2 dense children is itself dense. Therefore
  $r^{(1)}(v_j),\ldots,r^{(\lfloor\tau/2\rfloor)}(v_j)$ are all dense.
  Therefore, $w=r^{(\kappa)}(v_j)$ for some $\kappa >
  \lfloor\tau/2\rfloor$.  On the other hand, $u=r^{(\gamma)}(v_j)$ for
  some $\gamma\in\{\lfloor\tau/2\rfloor,\lfloor\tau/2\rfloor+1\}$.
  In either case $\kappa \ge \gamma$ so $w=u$ or
  $w=r^{(\kappa-\gamma)}(u)$, as required.
\end{proof}

%\section{Crap}
%
%Then $\|vw\|_{G-F}=\infty$ so $G-F$ is a not a $t$-spanner of $V\setminus F=V\setminus F^*$ for any $t<\infty$.}
%A slightly more sophisticated argument shows that any $(1+1/d)k$-robust $t$-spanner must have $\Omega(dn)$ edges.\footnote{Proof: If $G$ has $
%
%rules out the existence
%of $(k+o(k))$-robust spanners having a subquadratic number of edges.
%
%
%

\bibliographystyle{plain}
\bibliography{robust2}

\end{document}









