\documentclass{patmorin}
\usepackage{pat}
\usepackage{hyperref}
\usepackage{paralist}
\usepackage[noend]{algorithmic}
\hypersetup{colorlinks=true, linkcolor=linkblue,  anchorcolor=linkblue,
citecolor=linkblue, filecolor=linkblue, menucolor=linkblue,
urlcolor=linkblue, pdfcreator=Me, pdfproducer=Me} 
\setlength{\parskip}{1ex}

\title{\MakeUppercase{Robust Geometric Spanners}}

\author{Prosenjit Bose, Paz Carmi, Michael van Dyk, Pat Morin, 
  \newline
  Luis Fernando Schulz Xavier da Silveira,
  and Jan Volec}

\newcommand{\note}[2]{{\color{red}[#1:~#2]}}

\DeclareMathOperator{\sz}{size}
\DeclareMathOperator{\rank}{r}
\DeclareMathOperator{\diam}{diam}
\DeclareMathOperator{\lbl}{label}

\begin{document}
\maketitle


\begin{abstract}
  For any $\epsilon >0$ and any $n\in\N$, we construca a graph $G=(V,E)$
  with vertex set $V=\{0,\ldots,n-1\}$ having $O(\epsilon^{-2}n\log n)$
  edges, and such that, for any $F\subseteq V$, $G-F$ contains a monotone
  path of length at least $n-(1+\epsilon)|F|$.  The number of edges
  in $G$ matches a lower-bound of Bose \etal\ (2013) and improves the
  previous construction of Buchin, Hulshof and Ol\'ah (2018) which has
  $O(n^{1+\epsilon})$ edges and guarantees $|F^*|\le c^{1/\epsilon}|F|$,
  for some constant $c\ge 3$ and any $\epsilon >0$.
\end{abstract}

\section{Introduction}

A geometric graph $G=(V,E)$ with vertex set $V\subset\R^d$ is a (geometric)
$t$-spanner of a subset $X\subset V$ if, for every pair of vertices
$v,w\in X$, $v\neq w$,
\begin{equation}
	\frac{\|vw\|_G}{\|vw\|} \enspace , \eqlabel{spanning-ratio}
\end{equation}
where $\|vw\|$ denotes the Euclidean distance between $v$ and $w$ and
$\|vw\|_G$ denote the Euclidean length of the shortest path between $v$
and $w$ in $G$, where use the convention that $\|vw\|_G=\infty$ if $v$
and $w$ are in different components of $G$.  
Most of the research on spanners focuses on \emph{sparse}
spanners, where the number of edges in $G$ is linear, or close to linear,
in $|V|$.  In addition to having natural applications to transportation
networks, sparse $t$-spanners have found numerous applications in
approximation algorithms and geometric data structures.

For any non-decreasing function $f\colon\N\to\N$, Bose \etal\ say
that a geometric graph $G$ is an \emph{$f(k)$-robust $t$-spanner}
if, for every set $F\subset V(G)$, there exists a set $F^*\supseteq
F$ such that $|F^*|\le f(|F|)$ and the graph $G-F$ is a $t$-spanner
of $V(G)\setminus F^*$.  In networking applications, this definition
captures the idea that the number of nodes harmed by a set faulty nodes
should be bounded by a function of the number of faulty nodes.

Under this definition, the most robust spanner one could hope for
would be a $k$-robust spanner, but it is straightforward to argue
that, even for one dimensional point sets, the complete graph is the
only $k$-robust spanner.\footnote{Proof: Consider any pair of vertices
$v,w\in V$ that are not adjacent in $G$ and let $F=V\setminus\{v,w\}$.
Then $\|vw\|_{G-F}=\infty$ so $G-F$ is a not a $t$-spanner of $V\setminus
F=V\setminus F^*$ for any $t<\infty$.} The complete graph is not sparse,
and is therefore not suitable for many applications. A natural second-best
option is a $ck$-robust spanner with a near-linear number of edges,
for some $c>1$ .

\section{Background}

A weak $d$-expander for a pair of sets $(A,B)$ is a bipartite graph
$H=(A,B,E)$ in which, for every $B'\subset B$,
\[
	|N_H(B')| \ge \min\{(1-1/d)|A|, d|B'|\}
\]
It is well known that weak $d$-expanders exist that have $O(d|B|)$ edges.\note{PM}{Check the dependence on $d$.}

A strong $t$-expander for an ordered pair of sets $(A,B)$ is a bipartite graph $H=(A,B,E)$ in which, for every $A'\subset A$,
\[
	|N_H(A')| \ge \min\{t|A'|,(1-\sqrt{t})|B|\}
\]

It is well-known that strong $t$-expanders exist that have $O(\epsilon^{-1}(|A|+|B|))$
edges.


For a rooted binary tree $T$, $L(T)$ denotes the set of leaves in $T$. The
\emph{size} of $T$, denoted $|T|$ is the number of leaves $|L(T)|$
of $T$. For a node $u$ in $T$, $T_u$ denotes the subtree of $T$ rooted
at $u$.  We say that $T$ is \emph{full} if each non-leaf node of $T$
has exactly two children.

Define the \emph{rank} of $T$ as $\floor\log_2|T|$.  We say that a node
$u$ of $T$ is \emph{marked} in $T$ if $u$ is the root of $T$ or if $w$
is the paent of $u$ and $\rank(T_u)<\rank(T_w)$.  Note that if $T$
is full then, for any non-leaf node $u$ of $T$, there is a marked node
$u_0\neq u$ in $T_u$ such that $|T_{u_0}|> |T_u|/4$.



\section{The Construction}

$P\subset \R^d$ is an $n$-point set.  $T$ is a fair split tree for $P$,
so $L(T)=P$.

\subsection{Exploding into the Root}

Let $T$ be the fair-split tree for an $n$-point set $P$ and consider
the following recursively constructed graph $G_{T}$ whose vertex set
is $L(T)$.  If $|T| \le \kappa$ for some constant $\kappa$, then $G_T$ uses is the complete graph on $L(T)$.  

If $|T|>\kappa$, let $u_0$ be a node of $T$ with the property that
$|T|/3\le |T_{u_0}|\le 2|T|/3$.  Let $T_1$ be the full binary tree
obtained from $T_u-T_{u_0}$ by contracting an edge incident to the unique
non-leaf node of $T_u-T_{u_0}$ that has only one child.  The graph $G_{T}$
contains an expander $H_T=(L(T),E_T)$. This expander has parameters $d>1$,
$\alpha, \beta,\zeta,\eta > 0$ and is constructed so that it satisfies
the following properties:
\begin{enumerate}
	\item[(PR1)] For any $X\subset L(T_{u_0})$ with
  $|X|<(1-\beta\rank(T)/d)|T_{u_0}|$, the number of $p\in L(T_1)$ such
  that $N_{H_T}(p)\subset X$ is most $(\alpha/d)|X|$.

   \item[(PR2)] For any set $S\subset L(T_{u_0})$ with $|S|\ge (\zeta/d)L(T_{u_0})$, $|N_{H_T}(S)|\ge (1-\eta/d)|T|$.
\end{enumerate}
In our construction, $d=\Theta(\log^2 n)$ and the remaining parameters
are small values that are upper bounded by some function of $\epsilon$. In particular, for any constant $\epsilon >0$, these parameters are also constant.

\begin{clm}
	For any constants $\alpha,\beta,\zeta,\eta>0$, the graph $H_T$ has 
	$O(d^2|T|)$ edges.
\end{clm}

Finally, we recursively construct $G_{T_{u_0}}$ and $G_{T_1}$ and add the
edges of each of the resulting graphs to $G_{T}$. This concludes the description of the graph $G_T$.

\begin{clm}
  $G_{T}$ has $O(d^2|T|\log |T|)$ edges.
\end{clm}

\begin{proof}
  The graph $H_T$ has
  $O(d^2|T|)$ edges.  The recursive constructions are on two trees $T_{u_0}$
  and $T_1$ where $|T_{u_0}|+|T_1|=|T|$ and $\max\{|T_{u_0}|,T_1\}\le
  2|T|/3$. It follows that the depth of recursion is at most
  $\log_{3/2}|T|$ and each level of recursion contributes a total of
  $O(d^2|T|)$ edges for a total of $O(d^2|T|\log|T|)$ edges.
\end{proof}

Define $\rank(T)=\floor{\log_{3/2} |T|}$ and observe that, in
the preceding construction, $\rank(T_{u_0}) \le \rank(T)-1$ and
$\rank(T_1)\le\rank(T)-1$.  Let $F$ be an arbitrary subset of $P$.  We say
that $T$ is \emph{$F$-dense} if $|L(T)\cap F|\ge (1-\delta\rank(T)/d)|T|$
for some constant $\delta$ to be discussed shortly.  Define the set
$F^+_T$ as follows (here $u_0$ and $T_1$ are defined as above):

\begin{enumerate}
  \item If $T$ is $F$-dense, then $F^+_T=L(T)$.
  \item $F^+_T$ contains $F^+_{T_{u_0}}\cup F^+_{T_1}$.
  \item If $T_{u_0}$ is not $F$-dense, then $F^+_T$ contains every point $p\in L(T_1)$ such that $N_{H_T}(p)\subset F^+_{T_{u_0}}$.
\end{enumerate}


\begin{clm}
	$|F^+_T| \le (1+\epsilon\rank(T)/d)|F\cap L(T)|$.
\end{clm}

\begin{proof}
  The proof is by induction on $\rank(T)$. If $|T|=1$, the claim is
  obvious. For $|T|\ge 2$, there are two cases to consider:
  \begin{enumerate}
    \item $T$ is $F$-dense. Then  $|L(T)\cap F|\ge
    (1-\delta\rank(T)/d)|T|$.  So
     \[
       |F^+_T|=|T|
	  \le \frac{|L(T)\cap F|}{1-\delta\rank(T)/d} 
          \le (1+\epsilon\rank(T)/d)|L(T)\cap F|
     \]
     provided that $\epsilon \ge 1/(1-\delta)-1$.

    \item $T$ is not $F$-dense. Recall that
    $\rank(T_{u_0}),\rank(T_1)\le\rank(T)-1$ so, by induction,
     \[
        |F^+_{T_{u_0}}|	+ |F^+_{T_{1}}| 
          \le (1+\epsilon\rank(T)/d)|F\cap L(T)| - (\epsilon/d)|F\cap L(T)|
     \]
     All that remains is to consider the set $F^+_{01}$ of points in 
     $L(T_1)$ added to $F^+_T$ in Step~3.  In particular, we must show that
     $|F^+_{01}|\le (\epsilon/d)|F\cap L(T)|$. By Property~(P1) of $H_T$,
     \begin{align*}
      |F^+_{01}| \le (\alpha/d)|F^+_{T_{u_0}}|
        & \le (\alpha/d)(1+\epsilon\rank(T_{u_0})/d)|F\cap L(T_{u_0})| \\
        & \le (\alpha/d)(1+\epsilon\rank(T_{u_0})/d)|F\cap L(T)|\\
        & = (\alpha/d+\alpha\epsilon\rank(T_{u_0})/d^2)|F\cap L(T)| \\
        & = (\alpha/d+\alpha\epsilon/d)|F\cap L(T)| 
	     & \text{(for $d\ge \rank(T)$)} \\
	& \le (\epsilon/d)|F\cap L(T)| \enspace ,
   \end{align*}
   provided that $\alpha+\alpha\epsilon \le \epsilon$, i.e, 
   $\alpha \le 1/(1+1/\epsilon)$. Therefore, 
   $|F^+_T| = |F^+_{T_{u_0}}| + |F^+_{T_1}| + |F^+_{01}| \le (1+\epsilon\rank(T)/d)|F\cap L(T)|$, as required. \qedhere
   \end{enumerate}
\end{proof}


\begin{clm}
  For any $F\subset L(T)$ and every point $p\in L(T)\setminus F^+_T$,
	there exists $X\subset L(T)$, $|X|\ge (1-a\rank(T)/d)|T|-|F^+_T\cap L(T)|$
  such that for every $q\in X$, $G_T-F$
  contains a path from $p$ to $q$ of length $C\diam(T)$.
\end{clm}

\begin{proof}
  The proof is by induction on $|T|$.  If $|T|\le\kappa$, the result is trivial since $G_T$ is the complete graph.
  For $|T|>\kappa$, there are several cases to consider:
  \begin{enumerate}
	  \item $T_{u_1}$ is $F$-dense. In this case $F^+_T\supseteq
	  F^+_{T_{u_1}}= L(T_{u_1})$. Therefore, $p\in L(T_1)$.
          Now, we apply induction on $T_1$
       and obtain a set $X$ that is $p$-reachable with paths of length at
		  most $C\diam'(T_1)< C\diam'(T)$.  Now, 
	\begin{align*}
		|X| & \ge (1-a\rank(T_1)/d)|T_1|-|F^+_{T_1} \cap L(T_1)| \\
		& \ge (1-a\rank(T)/d)|T_1|-|F^+_{T_1} \cap L(T_1)| 
	 	& \text{(since $\rank(T)>\rank(T_1)$)} \\
		& \ge (1-a\rank(T)/d)|T_1|-|F^+_{T} \cap L(T_1)| 
		& \text{(since $F^+_T\supseteq F^+_{T_1}$)} \\
		& > (1-a\rank(T)/d)|T_1|-|F^+_{T} \cap L(T_1)| 
		  + (1-a\rank(T)/d)|T_{u_0}|-|F^+_{T_{u_0}}|
		& \text{(since $|T_{u_0}|=|F^+_{T_{u_0}}|$)} \\
		& = (1-a\rank(T)/d)|T|-|F^+_{T}\cap L(T_1)| - |F^+_{T_{u_0}}|
		& \text{(since $|T_1|+|T_{u_0}|=|T|$)} \\
		& = (1-a\rank(T)/d)|T|-|F^+_{T}\cap L(T_1)| - |F^+_T\cap L(T_{u_0})|
		& \text{(since $F^+_{T_{u_0}}=F^+_T\cap L(T_{u_0})$)} \\
		& = (1-a\rank(T)/d)|T|-|F^+_{T}| \enspace , 
	\end{align*}
	as required.
    \item $T_{u_1}$ is not $F$-dense. In this case, there are two subcases
    to consider:
    \begin{enumerate}
	    \item $p\in L(T_{u_0})$. 
		    Since $u_0$ is not the root of $T$, $\diam'(T_{u_0}) \le (1-1/2d)\diam'(T)$.  We can therefore apply induction on $T_{u_0}$ to find a $p$-reachable set $X_0\subseteq L(T_{u_0})$ of size 
\begin{align*}
	|X_0| & \ge (1-a\rank(T_{u_0})/d)|T_{u_0}|-F^+_{T_{u_0}} \\
	      & \ge (1-a\rank(T_{u_0})/d)|T_{u_0}|-(1-\epsilon\rank(T_{u_0})/d)|T_{u_0}| \\
	      & = (\epsilon-a)(\rank(T_{u_0}/d)|T_{u_0}| \\
	      & \ge (\epsilon-a)((\rank(T)-3)/d)|T_{u_0}| \\
	      & \ge ((\epsilon-a)/d)|T_{u_0}| 
		& \text{(for $|T|\ge \kappa \ge 5$)} \\
	      & \ge ((\epsilon-a)/(3d))|T| \enspace .
\end{align*}
		    By Property~PR2 of $H_T$ (with $\zeta = (\epsilon-a)/3$ and $\eta \le a\rank(T)$), we can then take $X=N_{H_T}(X_0)\setminus F^+_T$.  Then $|X| \ge (1-a\rank(T)/d)|T|-|F^+_T|$ and every point $q\in X$ is reachable from $p$ by a path of length at most
		    \[ (C(1-(1/2d))+1)\diam'(P) \le C\diam'(P) \]
	     for $C\ge 4d$.

	    \item $p\in L(T_1)$. Since $p\not\in F^+_T$, $H_T$ contains an
	    edge from $p$ to some point $p'\in L(T_{u_1})$.  As described
	    in the previous case, there is a $p'$-reachable set $X$ of
	    size $(1-a\rank(T)/d)|T|-|F^+_T|$ that is reachable from $p'$
	    by paths of length at most $((1-1/2d)C+1)\diam'(P)$.
	    The edge $pp'$ has length at most
	    $\diam'(T)$. Therefore $X$ is $p$-reachable using
	    paths of length at most $((1-1/2d)C+2)\diam'(P)
	    \le C\diam'(P)$ for $C\ge 4d$.
    \end{enumerate}
  \end{enumerate}
\end{proof}

\subsection{Multiple Scales}

For each node $u$ of $T$, define
$\lbl(u)=\floor{\log_{1+\epsilon}|T_u|}$. We say that a node $u$ of $T$
is \emph{special} if $u$ is a leaf or if $\lbl(u)$ is different from
both its children.  If $u$ is special, then $T_u$ is also special.  Observe that for every node $w$ of $T$, $T_w$
contains a special subtree $T_u$ with $|T_u|\ge (1-O(\epsilon))|T_w|$.
\note{PM}{Prove this, or redefine special appropriately if it's not
true.}

\begin{lem}
  For any constant $\epsilon' >0$, any $n\in\N$, and any $n$-point
  set $P\subset\R^d$ with fair-split tree $T$, there exists a graph
  $G_P=(P,E)$ with $O(n\log^5 n)$ edges such that, for any $F\subseteq
  P$, there exists a superset $F^+_P\supseteq F$ with $|F^+_P|\le
  (1+\epsilon')|F|$ such that for any node $w$ of $T$ and any point
  $p\in L(T_w)\setminus F^+_P$, there is a special node $u$ in $T_w$
  and a subset $X\subseteq L(T_u)$
  with $|X|\ge \epsilon')|T_u|$ such that for every $q\in X$, $G_P-F$
  contains a path from $p$ to $q$ of length at most $(C+1)\diam'(T_w)$.
\end{lem}


\begin{proof}
  Let $\epsilon >0$ be some constant that is upper bounded by some
  function of $\epsilon'$.  The graph $G_P$ contains all edges of
  $G_{T_u}$ for each special node $u$.  The parameter $d$ in the
  construction of $G_{T_u}$ is set to $d=c\log^2 n$ for some sufficiently
  large constant $c$.  Therefore, the graph $G_{T_u}$ has $O(|T_u|\log^4
  n)$ edges.  For any two special nodes, $u$ and $u'$ with $\lbl(u)=\lbl(u')$,
  the subtrees $T_u$ and $T_{u'}$ are disjoint.  This implies that the
  total number of edges in all graphs generated by nodes with label
  $i$ is $O(n\log^4 n)$.  Since there are only $O(\log n)$ different
  label values, it follows that the total number of edges in $G_P$
  is $O(n\log^5 n)$.

  We say that a node $w$ with parent $x$ in $T$ is \emph{left out of
  $u$} if $x$ is not special and the largest special subtree $T_u$
  in $T_x$ is not contained in $T_w$.  Note that each special node is
  the smaller of the two children of $x$, so that any root to leaf
  path in $T$ contains at most $\log_2 n$ left out nodes.  
	
  For a special node $u$, let $w_1,\ldots,w_k$ be the nodes left out of $u$, and  let $W_u=\bigcup_{i=1}^k L(w_i)$.  For each special node $u$ we construct
  an expander $H_u$ for the pair
  $(W_u,L(T_u))$. This expander has the property that, for any
  subset $X\subset L(T_u)$, $|X|\le (1-a)|T_u|$, the number of $p\in W_u$
  with $N_{H_u}(p)\subset X$ is at most $\epsilon|X|/\log_2 n$.
  \note{PM}{Lots of overlap here with the construction from the previous
  section. Can we unify the two?}  The graph $H_u$ has $O(|W_u|\log^2
  n)$ edges and by summing over all left out nodes $w$ these graphs have
  a total of $O(n\log^3 n)$ edges.  The graph $G_P$ contains $H_w$
  for each left out node $w$.

  This concludes the description of $G_P$
  and the analysis of the number of edges in $G_P$. What remains it is
  to describe and analyze the set $F^+_P$.

  For any subset $F\subset P$, the set $F^+_{T_u}$ is defined for every
  special node $u$ of $T$.  We define
  \[  F^*_P = \cup\{F^+_{T_u}: \text{$u$ is a special node of $T$}\}
  \]
  For each special node $u$, $|F^+_{T_u}| \le (1+O(\epsilon\log
  n/d))|F\cap L(T_u)|$.  It follows that $|F^*_P| \le (1+\epsilon)|F|$
  for a sufficiently large constant $c$.

  Next, we define the set $F^{**}_P$ as follows: For each special node $u$
  and each point $p\in W_u$, we include $p$ in $F^{**}_P$ if $N_{H_u}(w)
  \subseteq F^+_{T_u}$.  Note that, in this step, each special node $u$
  contributes at most $\epsilon |F^+_u|/\log n$ points to $F^{**}_P$
  and therefore $|F^{**}_P| \le (1+O(\epsilon))|F|$.

  Next we define 
  \[  F^{***}_P = \cup\{L(T_u) : |(F^*_P\cup F^{**}_P)\cap L(T_u)| \ge (1-\epsilon)|T_u|\} \enspace . \]
  Again, it is easily verified that $|F^{***}_P| \le (1+O(\epsilon))|F|$.

  Finally, we define $F^+_P= F^*_P\cup F^**_P\cup F^{***}_P$.   This
  concludes the definition of $F^+_P$ and the analysis of its size.
  Note that by choosing $\epsilon = \epsilon' /c$ for a sufficiently large
  constant $c$, $|F^+_P|\le (1+\epsilon')|F|$, as required by the theorem.
  All that remains is to prove that $F^+P$ satisfies the conditions of
  the theorem.  

  Now, consider any node $w$ of $T$ and any point $p\in L(T_w)\setminus F^+_P$.
  Let $u$ be a special node in $T_w$ such that $T_u$ is maximum.
  There are two cases to consider:
  \begin{enumerate}
    \item $p\in L(T_u)$. In this case, there is a subset $X\subset L(T(u))$
     such that, for each node $q\in X$, $G_{T_u}$ contains a path from
     $p$ to $q$ of length at most $C\diam(T_u)$. Furthermore,
    \begin{align*}
	    |X| & \ge (1-a/\log n)|T_u| - F^+_{T_u} \\
	        & \ge (1-a/\log n)|T_u| - (1-\epsilon)|T_u| \\
	       & \text{since $F^+_{T_u} \subset F^*_P$ and $|F^*_P\cap T_u|\le (1-\epsilon)|T_u|$} \\
		& \ge \Omega(\epsilon)|T_u| \\ 
		& \ge \Omega(\epsilon)(1-\epsilon)|T_w|  \\
		& \ge \Omega(\epsilon)|T_w| 
    \end{align*}
    as required.
    \item $p\in L(T_w)\setminus L(T_u)$.  In this case, since $p\not\in
    F^{**}_P$, $G_P-F$ contains an edge $pp'$ with $p'\in L(T_u)\setminus
    F^{+}_{T_u}$.  The edge $pp'$ has length at most $\diam(T_x)$.
    We can now proceed, as in the previous case, from $p'$.
  \end{enumerate}
\end{proof}


%    \begin{enumerate}
%	    \item If $p\in L(T_{u_0})$, then we can apply induction on $T_{u_0}$ to obtain a set $X_0$ that is $p$-reachable by paths of length at most $C\diam'(T_{u_0}) < (1-1/2d)C\diam'(T)$.  The set reached this way has size $(1-\beta\diam(X)$.      Hmmmmmmmmm
%
%       \item If $p\in L(T_1)$, then there is an edge from $p$ to some $p'\in L(T_{u_0}\setminus F^+_{T_{u_0}}$.  We can then proceed from $p'$ as in the previous case to obtain a set $X$ reachable by paths of length at most $(1-1/2d)(C+2)\diam(T)\le C\diam(T)$ for $C\ge XXX$.
%	    
%	    
%	    TFor any node $u$ of $T$, let $\diam'(u)$ denote the the sum of the lengths of the sides of the minimum bounding box of $u$.  Recall
%  \end{enumerate}
%
%  For a node $u$ of $T$, let $\diam'(u)$ denote 
%  If $p$ is in $L(T_{u_0})$ then $T_{u_0}$ is not $F$-dense, so we apply
%  induction on $T_{u_0}$ to get to a huge set in $L(T_{u_0})$. From
%  there we expand into $L(T)$ using Property~2.
%
%  If $p$ is in $L(T_1)$ then, since $p\not\in F^+_T$, there is an edge
%  from $p$ to some point $p'$ in $L(T_{u_0})$.  At this point we proceed
%  from $p'$ as described above.
%\end{proof}
%

\bibliographystyle{plain}
\bibliography{robust2}

\end{document}









