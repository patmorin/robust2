\documentclass{patmorin}
\usepackage{pat}
\usepackage{hyperref}
\usepackage[noend]{algorithmic}
\hypersetup{colorlinks=true, linkcolor=linkblue,  anchorcolor=linkblue,
citecolor=linkblue, filecolor=linkblue, menucolor=linkblue,
urlcolor=linkblue, pdfcreator=Me, pdfproducer=Me} 
\setlength{\parskip}{1ex}

\title{\MakeUppercase{Optimal Robust Spanners in 1D}}

\author{Prosenjit Bose, Paz Carmi, Michael van Dyk, and Pat Morin}

\newcommand{\note}[2]{{\color{red}[#1:~#2]}}

\begin{document}
\maketitle


\begin{abstract}
  For any $\epsilon >0$ and any $n\in\N$, we construct a graph $G=(V,E)$
  with vertex set $V=\{0,\ldots,n-1\}$ having $O(n\log n)$ edges, and such
  that, for any set $F\subset V$, $G-F$ contains a monotone
  path of length at least $n-(1+\epsilon)|F|$.  The number of edges
  in $G$ matches a lower-bound of Bose \etal\ (2013) and improves the
  previous construction of Buchin, Hulshof and Ol\'ah (2018) which has
  $O(n^{1+\epsilon})$ edges and guarantees $|F^*|\le c^{1/\epsilon}|F|$,
  for some constant $c\ge 3$ and any $\epsilon >0$.
\end{abstract}

\section{Introduction}

\section{The Construction}

%For any integers $i<j$, let $[i,j]=\{i,\ldots,j-1\}$.

Without loss of generality, assume $n$ is a power of 2.  Construct a
complete ordered binary tree $T$ of height $\log_2 n$ whose nodes are
contiguous sets of integers..  The leaves of $T$, in left-to-right order,
are the singleton sets $\{0\},\ldots,\{n-1\}$ and each internal node of
$T$ is the set obtained by taking the union of its two children.  Stated
another way, $T=T(0,\log_2 n-1)$, whree $T(i,b)$ has a single node $\{i\}$
if $b=0$. Otherwise $T(i,b)$ has a root node $\{i,i+1,\ldots,i+2^{b-1}-1\}$
whose two children are the subtrees $T(i,b-1)$ and $T(i+2^{b-1}, b-1)$.

Note that the nodes of $T$ can be partitioned in levels $L_0,\ldots,L_{\log_2 n}$, where each $L_i=\{u_{i,0},\ldots,u_{i,2^i-1}\}$ and
\[
	u_{i,k} = \{kn/2^i,\ldots,(k+1)n/2^{i}-1\} \enspace .
\]
For each consecutive pair $(u_{i,k},u_{i,k+1})$ of nodes in $L_i$ we say
that $u_{i,k}$ and $u_{i,k+1}$ are \emph{buddies} and we call $u_{i,k+1}$
the \emph{right buddy} of $u_{i,k}$.

\subsection{A First Try}

We now describe a first construction of $G$ that has $O(\epsilon^{-1}n\log
n)$ edges and guarantees that, for any $F\subset V$,  $G-F$ has a monotone
path of length at least $n-(2+4\epsilon)|F|$.

The graph $G$ contains, for each $i\in\{1,\ldots,\log_2 n\}$ and each
$k\in\{0,\ldots,2^{i}-2\}$ the edges of a bipartite \emph{expander graph}
$B_{i,k}$.  The vertex set of $B_{i,k}$ has two classes $A=u_{i,k}$ and
$B=u_{i,k+1}$ and has the following property:  For any subset $S\subset
A$ with $|S|\ge \epsilon |A|$, and every vertex $y$ in $B$, there is
some $x\in A$ such that $xy$ is an edge of $B_{i,k}$. It is known that
such expanders exist in which every node has degree $O(\epsilon^{-1})$
\cite{X}.  Observe that for every pair of buddies $(u,u')$ in $T$,
there is an expander graph between $u$ and $u'$.

Each node $u_{i,k}$ of $T$ is involved in at most two expander
graphs $B_{i,k-1}$ and $B_{i,k}$ and each such graph contributes
$O(\epsilon^{-1}n/2^{i})$ edges to $G$.  Therefore, each level $L_i$
contributes at most $O(\epsilon^{-1} n)$ edges to $G$ for a total of
$O(\epsilon^{-1}n\log n)$ edges.  

Next we define our set $F^*\supseteq F$.  Given a set $F\subseteq V$,
we say that a node $u$ of $T$ is \emph{sparse} (with respect to $F$)
if $|u\cap F| \ge (1-\epsilon)|u|$ and $u$ is \emph{dense} (with respect
to $F$) otherwise.  We then define
\[
	F^* = \cup\{ u_{i,k}\in V(T): \text{$u_{i,k}$ is dense or $u_{i,k+1}$ is dense} \} \enspace .
\]
That is, a node $u$ of $T$ contributes to $F^*$ if $u$ is dense or $u$'s
right buddy is dense.

First we show that $F^*\supseteq F$ is not overly large:

\begin{clm}\clmlabel{size}
  $F^*\supseteq F$ and, for any $\epsilon < 1/2$, $|F^*|\le (2+4\epsilon)|F|$.
\end{clm}

\begin{proof}
  Consider the set $M$ of \emph{maximal} dense nodes containing exactly
  those nodes $u$ of $T$ that are dense and such that no ancestor of $u$
  is dense.  Observe that every $x\in F$ appears in exactly one node
  of $M$.  This immediately implies that $F^*\supseteq F$.

  To upper bound $|F^*|$, observe that $|F^*| = \sum_{u\in M} 2|u|$
  and that 
	$|F| \ge \sum_{u\in M} (1-\epsilon)|u|$, 
	so
  \[
    \frac{|F^*|}{|F|} 
      \le \frac{2}{1-\epsilon} 
	= 2\sum_{r=0}^\infty\epsilon^r 
	= 2(1+\epsilon + \epsilon\sum_{r=1}^\infty \epsilon^r 
	\le 2(1+2\epsilon) \enspace . \qedhere
	  \]
\end{proof}

Next we show that $G-F$ contains a monotone path of length $n-|F^*|$.

\begin{clm}\clmlabel{path}
  For any $x\in V\setminus F^*$ and any $y\in \{x+1,\ldots,n\}\setminus
  F$, $G-F$ contains a monotone path from $x$ to $y$.
\end{clm}

\begin{proof}
  Let $v_0,\ldots,v_{\log_2 n}$ be the path from the leaf of $T$
  containing $x$ to the root of $T$ and let $v_t$ be the last node on
  this path that is the left child of its parent.  Observe that, for
  each $j\in\{0,\ldots,t\}$, $v_j$ has a right neighbour $v_j'$ and,
  since $x\not\in F^*$, $v_j'$ is sparse.

  We claim that, for each $j\in\{0,\ldots,k\}$ and each node $y\in
  v_j'\setminus F$, $G-F$ contains a monotone path from $x$ to $y$.
  We prove this claim by induction on $j$.  For the base case, $v_0=\{x\}$
  and $v_0'$ are buddy leaves in $T$ and there is an expander graph
  for $(v_0,v_0')$
  that contains the edge $xy$.

  For $j>1$, the inductive hypothesis states that $G-F$ contains
  a monotone path from $x$ to each $x'\in v_{j-1}'\setminus F$.
  Let $S=v_{j-1}'\setminus F$.  There are two cases to consider:
  \begin{enumerate}
    \item $v_{j-1}$ is the left child $v_{j}$.  In this case,
	  $S$ is a subset of $v_j$ of size at least $\epsilon|v_{j-1}|
	  = \epsilon|v_j|/2$.  The graph $G$ contains an expander graph
	  for the buddy pair $(v_j,v_j')$ and therefore every $y\in v_j'$
	  contains a neighbour $x'\in S$.  The (inductive) path from
	  $x$ to $x'$ plus the edge $x'y$ is a monotone path in $G-F$
	  from $x$ to $y$, as required.

    \item $v_{j-1}$ is the right child of $v_j$.	In this case, let
	  $v_{j-1}''$ be the right neighbour of $v_{j-1}'$.
	  The graph $G$ contains an expander graph for the pair
	  $(v_{j-1}',v_{j-1}'')$ and $S$ is a subset of $v_{j-1}'$ of
	  size at least $\epsilon|v_{j-1}'|$ so every $y\in v_{j-1}''$ is
	  adjacent to some $x'\in v_{j-1}'$.  Therefore, for every $y\in
	  v_{j-1}''$, there is a monotone path from $x$ to $x'$ to $y$.
	  We finish by observing that $v_j' = v_{j-1}'\cup v_{j-1}''$.
  \end{enumerate}
  Finally, we observe that, for any $y\ge x$, $y$ is contained in at least one of of $v_0',\ldots,v_t'$.  Therefore $G-F$ contains a monotone path from $x$ to each $y\in\{x+1,\ldots,n\}\setminus F$, as required.
\end{proof}

\begin{prop}
	For the graph $G$ defined above and any $F\subset V(G)$, $G-F$ contains a monotone path of size at least $n-(2+4\epsilon)|F|$.
\end{prop}

\begin{proof}
	Let $x_1<x_2<\cdots<x_m$ be the elements of $V(G)\setminus F^*$.
	Then, by \clmref{path}, there is a monotone path from $x_i$
	to $x_{i+1}$, for each $i\in\{1,\ldots,m-1\}$.
	The concatentation of all these paths is a monotone path of size $n-|F^*| \ge n-(2+4\epsilon)|F|$, by \clmref{size}.
\end{proof}

\section{Achieving $1+\epsilon$}


Next, we show how to improve our construction to obtain a set $F^*$
of size at most $(1+\epsilon)|F|$.  To achieve this, we add more more
expander graphs. For each $u$ of $T$, let $r(u)$ denote $u$'s right buddy.
Our graph $G$ is parameterized by an natural number $q$.  For each node
$u$ of $T$, and each $i\in\{1,\ldots,q\}$, $G$ contains the edges of an
expander graph for the pair $(u,r^{(i)}(u))$.  Therefore, the graph $G$
contains $O(q\epsilon^{-1} n\log n)$ edges.






\bibliographystyle{plain}
\bibliography{robust2}

\end{document}









