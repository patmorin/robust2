\documentclass{patmorin}
\usepackage{pat}
\usepackage{hyperref}
\usepackage{paralist}
\usepackage{graphicx}
\hypersetup{colorlinks=true, linkcolor=linkblue,  anchorcolor=linkblue,
citecolor=linkblue, filecolor=linkblue, menucolor=linkblue,
urlcolor=linkblue, pdfcreator=Me, pdfproducer=Me} 
\setlength{\parskip}{1ex}

\title{\MakeUppercase{Optimal} $O(k)$-\MakeUppercase{Robust Spanners in One Dimension}}

\author{TBD}
%\author{Prosenjit Bose, Paz Carmi, Michael van Dyk, Pat Morin, 
%  \newline
%  Luis Fernando Schulz Xavier da Silveira,
%  and Jan Volec}

\newcommand{\note}[2]{{\color{red}[#1:~#2]}}

\begin{document}
\maketitle


\begin{abstract}
  For any $\epsilon >0$ and any $n\in\N$, we construct a graph $G=(V,E)$
  with vertex set $V=\{0,\ldots,n-1\}$ having $O(\epsilon^{-3}n\log n)$
  edges, and such that, for any $F\subseteq V$, $G-F$ contains a monotone
  path of length at least $n-(1+\epsilon)|F|$.  For any constant $\epsilon
  >0$, the number of edges in $G$ matches a lower-bound of Bose \etal\
  (2013) and improves the previous construction of Buchin, Hulshof and
  Ol\'ah (2018) which has $O(n^{1+1/\ell})$ edges and guarantees a
  monotone path of length $n- c^{\ell}|F|$, for some constant
  $c\ge 3$ and any $\ell\in\N$.
\end{abstract}

\section{Introduction}

A geometric graph $G=(V,E)$ with vertex set $V\subset\R^d$ is a (geometric)
$t$-spanner of a subset $X\subset V$ if, for every pair of vertices
$v,w\in X$, $v\neq w$,
\begin{equation}
  \frac{\|vw\|_G}{\|vw\|} \enspace , \eqlabel{spanning-ratio}
\end{equation}
where $\|vw\|$ denotes the Euclidean distance between $v$ and $w$ and
$\|vw\|_G$ denote the Euclidean length of the shortest path between $v$
and $w$ in $G$, where use the convention that $\|vw\|_G=\infty$ if $v$
and $w$ are in different components of $G$.  
Most of the research on spanners focuses on \emph{sparse}
spanners, where the number of edges in $G$ is linear, or close to linear,
in $|V|$.  In addition to having natural applications to transportation
networks, sparse $t$-spanners have found numerous applications in
approximation algorithms and geometric data structures.

For any non-decreasing function $f\colon\N\to\N$, Bose \etal\ say
that a geometric graph $G$ is an \emph{$f(k)$-robust $t$-spanner}
if, for every set $F\subset V(G)$, there exists a set $F^*\supseteq
F$ such that $|F^*|\le f(|F|)$ and the graph $G-F$ is a $t$-spanner
of $V(G)\setminus F^*$.  In networking applications, this definition
captures the idea that the number of nodes harmed by a set faulty nodes
should be bounded by a function of the number of faulty nodes.

Under this definition, the most robust spanner one could hope for
would be a $k$-robust spanner, but it is straightforward to argue
that, even for one dimensional point sets, the complete graph is the
only $k$-robust spanner.\footnote{Proof: Consider any pair of vertices
$v,w\in V$ that are not adjacent in $G$ and let $F=V\setminus\{v,w\}$.
Then $\|vw\|_{G-F}=\infty$ so $G-F$ is a not a $t$-spanner of $V\setminus
F=V\setminus F^*$ for any $t<\infty$.} The complete graph is not sparse,
and is therefore not suitable for many applications. A natural second-best
option is a $ck$-robust spanner with a near-linear number of edges,
for some $c>1$ .

Bose \etal \cite[Theorem~3]{bose.dujmovic.ea:robust} show that for any
constants $c,t>1$ and any $n$-element set $V\subset\R$, any $ck$-robust
$t$-spanner $G=(V,E)$ must have $\Omega(n\log n)$ edges.  Their work
does not give any construction matching this lower bound, leaving a
gap in our understanding of robust spanners, even in one dimension.
Buchin, Hulshof and Ol\'ah \cite{buchin.hulshof.olah:robust} made the
first progress in filling this gap by constructing, for any $\ell\in\N$,
a $k2^{O(\ell)}$-robust 1-spanner having $O(n^{1+1/\ell})$
edges.\footnote{As they describe it, the construction of Buchin \etal\
\cite{buchin.hulshof.olah:robust} is a $6^\ell k$-robust.  However,
using expanders, the base $6$ can be reduced to $3+\delta$ for
any $\delta >0$ while only increasing the number of edges by a factor
of $(1/\delta)^2$.  This is done by replacing their matchings with the
exploder graphs described in \thmref{exploder}.}  In the current paper,
we prove the following result:

\begin{thm}\thmlabel{main}
  For every $n$-element set $V\subset\R$ and every $\epsilon>0$,
  there exist a $(1+\epsilon)k$-robust 1-spanner $G=(V,E)$ with
  $O(\epsilon^{-3}n\log n)$ edges.
\end{thm}

The lower-bound discussed above implies that the number of edges
in \thmref{main} is optimal, except possibly for the dependence on
$\epsilon$.

In any graph $G$ whose vertex set is a set of real numbers, a path
$x_1,\ldots,x_m$ in $G$ is \emph{monotone} if $x_1<\cdots<x_m$.
The statement ``$G$ is a 1-spanner'' is equivalent to the statement
``there is a monotone path that includes all vertices of $G$.'' Therefore,
to prove \thmref{main} it is sufficient to show that we can construct a
graph $G=(V,E)$ with $V=\{0,\ldots,n-1\}$, having $O(\epsilon^{-3} n\log n)$
and such that, for any $F\subseteq V$, $G-F$ contains a monotone path
of size at least $n-(1+\epsilon)|F|$.  From this point on, we will focus
on this problem.

\section{The Construction}

Without loss of generality, assume $n$ is a power of 2 and refer to
\figref{sets}.  Construct a complete ordered binary tree $T$ of height
$\log_2 n$ whose nodes are contiguous sets of integers.  The leaves of
$T$, in left-to-right order, are the singleton sets $\{0\},\ldots,\{n-1\}$
and each internal node of $T$ is the set obtained by taking the union
of its two children.  Stated another way, $T=T(0,\log_2 n)$, where
$T(i,b)$ has a single node $\{i\}$ if $b=0$. Otherwise $T(i,b)$ has
a root node $\{i,i+1,\ldots,i+2^{b}-1\}$ whose two children are the
subtrees $T(i,b-1)$ and $T(i+2^{b-1}, b-1)$.

\begin{figure}
  \centering{\includegraphics[width=.98\textwidth]{figs/binary-tree-1}}
  \caption{The binary tree $T$ on $\{0,\ldots,n-1\}$.}
  \figlabel{sets}
\end{figure}

Note that the nodes of $T$ can be partitioned in levels $L_0,\ldots,L_{\log_2 n}$, where each $L_i=\{u_{i,0},\ldots,u_{i,2^i-1}\}$ and
\[
  u_{i,k} = \{kn/2^i,\ldots,(k+1)n/2^{i}-1\} \enspace .
\]
For each consecutive pair $(u_{i,k},u_{i,k+1})$ of nodes in $L_i$ we say
that $u_{i,k}$ and $u_{i,k+1}$ are \emph{buddies} and we call $u_{i,k+1}$
the \emph{right buddy} of $u_{i,k}$.  For any $k\in\{0,\ldots,2^{i}-1\}$,
we use the notation $r(u_{i,k})$ to denote the right-buddy, $u_{i,k+1}$,
of $u_{i,k}$.  We also use an iterated for version of this notation:
$r^{(0)}(u)=u$ and, for any $j\in\N$, $r^{(j)}(u)=r(r^{{(j-1)}}(u))$.

\subsection{Expanders}

Our construction $G$ will consist of the union of many \emph{exploder
graphs} whose properties are given by the following definition:

\begin{defn}
   A bipartite graph $B=(X,Y,E)$ with $|X|=|Y|=n$ is called an
   \emph{order-$n$ $\epsilon$-exploder} for $(X,Y)$ if, for
   every $X'\subseteq X$, $|X'|\ge \epsilon n/4$, $|N_B(X')| \ge
   (1-\epsilon/2)n$.
\end{defn}

Exploders are, of course, closely related to expanders \cite{hoory.linial.wigderson:expanders}.  However, where work on expanders typically focuses on ensuring that small subsets $X'$ expand to a subset $Y'$ that is a constant factor larger than $X'$.  With exploders, we only care about large subsets $X'$ but we wish to ensure that these subsets expand to a subset $Y'$ that contains a very large fraction of $Y$.

The proof of the following theorem is standard (see, e.g.,
\cite[Lemma~1.9]{hoory.linial.wigderson:expanders}).  We include it
only only for completeness, because it is short and it establishes the
dependence between the number of edges and $\epsilon$.

\begin{thm}\thmlabel{exploder}
   For any $\epsilon >0$ and any $n\in\N$, there exists
   an order-$n$ $\epsilon$-exploder having $O(n/\epsilon^2)$ edges.
\end{thm}

\begin{proof}
   Consider the random bipartite graph $B=(X,Y,E)$ in which each
   vertex in $X$ chooses $d$ random elements of $Y$ as neighbours, with
   replacement.  Fix a subset $X'\subset X$ of size $\epsilon n/4$ and
   a subset $Y'\subseteq Y$ of size $(1-\epsilon/2)n$.  The probability
   that all elements of $X'$ choose all their neighbours in $Y'$ is
   $(1-\epsilon/2)^{d\epsilon n/4}$.  Applying the union bound over
   all choices of $X'$ and $Y'$ we get that the probability that $B$
   is not an $\epsilon$-exploder is at most
  \begin{align*}
    \binom{n}{\epsilon n/4}\binom{n}{(1-\epsilon/2) n}(1-\epsilon/2)^{d\epsilon n/4} & 
    \le \left(\frac{en}{\epsilon n/4}\right)^{\epsilon n/4}
        \left(\frac{en}{(1-\epsilon/2)n}\right)^{(1-\epsilon/2)n}
              (1-\epsilon/2)^{d\epsilon n/4} \\
    & = \left(\frac{e^{1-\epsilon/4}}{(\epsilon /4)^{\epsilon/4}(1-\epsilon/2)^{1-\epsilon/2}}\right)^{n}
              (1-\epsilon/2)^{d\epsilon n/4} \\
    & \le \left(2e^{1-\epsilon/4}\right)^{n}
              (1-\epsilon/2)^{d\epsilon n/4} \\
    & = \left(2e^{1-\epsilon/4}(1-\epsilon/2)^{d\epsilon /4}\right)^{n} \\
    & \approx \left(2e^{1-\epsilon/4-\epsilon^2d/8}\right)^{n} \\
    & < 1
  \end{align*}
  for $d>16/\epsilon^2$.  Therefore, we conclude that there exists an order-$n$ $\epsilon$-exploder with at most $16n/\epsilon^2$ edges.
\end{proof}

Our graph $G$ will contain many $\epsilon$-exploders for pairs $(u,w)$
where $u,w\in L_i$ are nodes at the same level, $i$, of $T$.  Given a
set $F\subseteq V$, we say that a node $u$ of $T$ is \emph{sparse}
with respect to $F$ if $|u\cap F| \le (1-\epsilon)|u|$ and $u$ is
\emph{dense} (with respect to $F$) otherwise.  In other words, if $u$
is sparse then $|u\setminus F| \ge \epsilon |u|$.

The following observation is a consequence of the definitions so far:

\begin{obs}\obslabel{extend}
  Let $n$, $G$, $F$, $T$, and $L_0,\ldots,L_{\log_2 n}$ be as defined above and
  and assume the following:
  \begin{compactenum}
     \item $x\in \{1,\ldots,n\}$ is a vertex and $u,w\in L_i$ 
      is a pair of nodes of $T$;
     \item $\{x\} \prec u\prec w$;
     \item $G$ contains an $\epsilon$-exploder, $B_{uw}$, for $(u,w)$; and
     \item $u'\subseteq u$, $|u'|\ge\epsilon|u|/4$, and, for every $x'\in u'$, $G-F$
   contains a monotone path from $x$ to $x'$.
   \end{compactenum}
   Then there exists a
  subset $w'\subseteq w$, $|w'|\ge(1-\epsilon/2)|u|-|F\cap w|$, and, for
   every $y\in w'$, $G-F$ contains a monotone path from $x$ to $y$.
\end{obs}

\begin{proof}
  By the definition of $\epsilon$-exploder, $N_{B_{uw}}(u')$ is a
  subset of $w$ of size at least $(1-\epsilon/2)|w|$. Therefore
  $w'=N_{B_{uw}}(u')\setminus F$ has size $|w'|\ge
  (1-\epsilon/2)|w|-|F\cap w|$.  Every vertex $y$ in $w'$ has a neighbour
  $x'\in u'$. Appending the edge $x'y$ to the monotone path from $x$
  to $x'$ gives a monotone path from $x$ to $y$.
\end{proof}
   
  
\subsection{A $(2+\epsilon)k$-Robust Construction}

We now describe a first construction of $G$ that has $O(\epsilon^{-2}n\log
n)$ edges and guarantees that, for any $F\subseteq V$,  $G-F$ has a
monotone path of length at least $n-(2+O(\epsilon))|F|$.  Changing
notation slightly, this means that, for any $\delta >0$, $G$ is a
$(2+\delta)k$-robust 1-spanner with $O(\delta^{-1}n\log n)$ edges.

The graph $G$ contains, for each pair $(u,w)$ of buddies in $T$, an
exploder graph $B_{uw}$ for the pair $(u,w)$.  The graph $B_{uw}$ has $O(\epsilon^{-2}|u|)$ edges. Therefore, for each level $L_i$, the buddy pairs in $L_i$ contribute $O(\epsilon^-2 n)$ edges to $G$.  Summing over all levels we see that $G$ has $O(\epsilon^{-2}n\log n)$ edges.

Next we define a set $F^+\supseteq F$ as follows: 
\[
  F^+ = \cup\{ u\in V(T): \text{$u$ is dense or $r(u)$ is dense}\} \enspace .
\]
Refer to \figref{dense-sparse}.
First we show that $F^+\supseteq F$ is not overly large:

\begin{figure}
  \centering{\includegraphics[width=.98\textwidth]{figs/binary-tree-2}}
  \caption{An example with $\epsilon=1/4$.  Elements in $F$ are denoted by $\times$, dense nodes are shaded blue, sparse nodes whose right buddy is dense are shaded in pink. $F^+=\{4,5,6,7,8,9,10,11,18,19,20,22,23,24,25,29,30\}$ includes every value in every shaded node.}
  \figlabel{dense-sparse}
\end{figure}

\begin{clm}\clmlabel{size}
  $F^+\supseteq F$ and, for any $\epsilon < 1/2$, $|F^+|\le (2+4\epsilon)|F|$.
\end{clm}

\begin{proof}
  Consider the set $M$ of \emph{maximal} dense nodes containing exactly
  those nodes $u$ of $T$ that are dense and such that no strict ancestor
  of $u$ is dense.  Observe that every $x\in F$ appears in exactly one
  node of $M$.  This immediately implies that $F^+\supseteq F$.

  To upper bound $|F^+|$, observe that $|F^+| \le \sum_{u\in M} 2|u|$
  and that 
  \[ 
     |F| = \sum_{u\in M} |F\cap u| \ge \sum_{u\in M} (1-\epsilon)|u| \enspace , 
  \]
  so
  \[
    \frac{|F^+|}{|F|} 
      \le \frac{2}{1-\epsilon} 
  = 2\sum_{r=0}^\infty\epsilon^r 
  = 2(1+\epsilon + \epsilon\sum_{r=1}^\infty \epsilon^r 
  \le 2(1+2\epsilon) \enspace . \qedhere
  \]
\end{proof}

\begin{clm}\clmlabel{two-plus-epsilon}
  For any $F\subset V(G)$, $G-F$ contains a monotone path 
  of size at least $n - (2+O(\epsilon))|F|$.
\end{clm}

\begin{proof}
  For a sparse node $u$ of $T$ with $x<\min(u)$ we say that a subset
  $u'\subset u$ is a \emph{witness set} for $(x,u)$ if $|u'|\ge
  (1-\epsilon/2)|u|-|F\cap u|$ and, for every $x'\in u$, $G-F$ contains
  a monotone path from $x$ to $x'$.

  We will construct the path in $G-F$ in a sequence of phases. At
  the beginning of phase $p$ we begin at a vertex $x_p\not\in F^+$
  and the phase will end when we are able to construct a monotone path to some
  vertex $y_p>x_p$, $y_p\not\in F^+$.  For the first phase, we begin with $x_1$
  as the first vertex not in $F^+$, i.e., $x_1=\min(V(G)\setminus F^+)$.
  For each subsequent phase $p$, the starting vertex $x_p=y_{p-1}$ found
  in the previous phase.

  The $p$th phase consists of a sequence of rounds. Refer to
  \figref{one-phase}.   At the beginning of the $i$th round we have a
  triple $(a_i,u_i,u_i')$ where $a_i$ is a node of $T$ containing $x_p$,
  $u_i=r(a_i)$, and $u_i'$ is a witness set for $(x_p,u_i)$.  To begin
  phase $p$ we set $a_0=\{x_p\}$, $u_0=r(a_0)=\{x_p+1\}$, and $u_0'=u_0$.
  Since $x_p\not\in F^+$, neither $a_0$ nor $u_0$ is dense.  Since $a_0$
  and $u_0$ are buddies, $G$ contains the exploder $B_{a_0u_0}$ which
  contains the single edge with endpoints $x_p$ and $x_p+1$.  Therefore,
  $u_0'=\{x_p+1\}$ is indeed a witness set for $(x_p,u_0)$.

  \begin{figure}
    \centering{\includegraphics[width=.98\textwidth]{figs/binary-tree-3}\\[2em]
       \includegraphics[width=.98\textwidth]{figs/binary-tree-4}}
    \caption{The nodes $a_i$, $u_i$, and $w_i$ visited during
      phase $p$. The node $a_0$ is the leaf containing $x_p$.}
      \figlabel{one-phase}
  \end{figure}

  Now consider a general round that begins with $(a_i,u_i,u_i')$.
  Since $x_p\in a_i$ and $x_p\not\in F^+$, each of $a_i$ and $u_i$ are
  sparse with respect to $F$. There are three cases to consider. These
  cases are not mutually exclusive, but when more than one applies,
  we consider the first applicable case.
  \begin{enumerate}
   \item $u_i'\setminus F^+$ is non-empty. In this case we take
    $y_p=\min(u_i'\setminus F^+)$. We have found a monotone
    path from $x_p$ to $y_p\not\in F^+$ so we complete the phase and
    being the next phase starting from $x_{p+1}=y_p$.

    \item $u_i$ contains $n-1$ (see round $i=4$ in \figref{one-phase}).  
    In this case, $r(u_i)$ does not exist.
    This is the final phase.  We take $y_p$ to be any element of $u_i'$.
    Since $y_p\in F^+$ we can not begin another phase and we terminate
    the entire procedure, having constructed a path that terminates at
    any node $y_p$.

    \item $a_i$ and $u_i$ are siblings with a common
    parent $a_{i+1}$ (see rounds $i=1,2$ in \figref{one-phase}).
    In this case, we set $u_{i+1}=r(a_{i+1})$.
    Notice that $u_i'\subseteq u_{i}\subset a_{i+1}$ and that
    $|u_i'|\ge \epsilon|u_i|/2 = \epsilon|a_{i+1}|/4$.  Therefore, by
    \obsref{extend}, $u_{i+1}'= N_{B_{a_{i+1}u_{i+1}}}(u')\setminus F$
    is a witness set for $(x_p,u_{i+1})$.

   \item Otherwise, the parent $a_{i+1}$ of $a_i$ and the parent $u_{i+1}$
   of $u_i$ are distinct (see rounds $i=0,3$ in \figref{one-phase}).
   In this case $u_i$ is the left child of $u_{i+1}$.  Let $w_i=r(u_i)$
   be the right child of $u_{i+1}$.  Then $G$ contains the exploder
   $B_{u_iw_i}$, so by \obsref{extend} $w_i'=N_{B_{uw}}(u')\setminus F$
   is a witness set for $(x_p,w)$.

   Now, for every $y\in u_{i+1}'=u_i'\cup w_i'$, $G-F$ contains a monotone
   path from $x$ to $y$ and
   \begin{align*}
     |u_{i+1}'| 
     & \ge |u_i'|+|w_i'| \\
     & \ge (1-\epsilon/2)|u_{i}| - |F\cap u_i| + (1-\epsilon/2)|w_i| - |F\cap w_i| \\
     &= (1-\epsilon/2)|u_{i+1}| - |F\cap u_{i+1}|  \enspace .
   \end{align*}
  Therefore $u_{i+1}'$ is a witness set for $(x_p,u_{i+1})$.
  \end{enumerate}

  This completes the description of phase $p$ that constructs a
  monotone path in $G-P$ that begins at $x_p$
  and ends at $y_p$.  We will now argue that the vast majority of
  vertices in $\{x_p+1,\ldots,y_p-1\}$ are in the set $F^+$.  More
  specifically, we will show that $|F^+\cap\{x_p+1,\ldots,y_p-1\}| \ge
  (1-O(\epsilon))(y_p-x_p-1)$.

  Consider the sequence $u_0,\ldots,u_k$.  This sequence begins at
  $u_0=\{x_p+1\}$ and each set in the sequence is twice as large as its
  predecessor.   If, for some node $u_i$ in this sequence $u_i\supset
  u_{i-1}$, then $u_{i-1}$ is the left child of $u_i$ in $T$ and
  $w_{i-1}$ is the right child of $u_i$.    We create a new sequence
  $v_0,\ldots,v_k$ where
  \[
    v_i = \begin{cases} 
      w_{i-1} & \text{if $i>0$ and $u_{i-1}\subset u_i$} \\
      u_i & \text{otherwise}
    \end{cases}
  \]
  See \figref{v-sequence}.
  The nodes in the modified sequence $v_0,\ldots,v_k$ form a partition
  of $\{\min(v_0),\ldots,\max(v_k)\}$ and each node $v_i$ is at most
  four times the size of $v_{i-1}$.

  \begin{figure}
    \centering{\includegraphics[width=.98\textwidth]{figs/binary-tree-5}}
    \caption{The sequence $v_1,\ldots,v_4$ derived from the sequence $u_1,\ldots,u_4$ in \figref{one-phase}.}
    \figlabel{v-sequence}
  \end{figure}

  In Case~3, above, the phase will terminate in round $i+1$ if
  $|u_{i+1}\setminus F^+|>\epsilon|u_{i+1}|/2$.  In Case~4, the phase
  will terminate in round $i+1$ if $|w_i\setminus F^+|>\epsilon|w_i|/2$.
  This implies that, for each $i\in\{1,\ldots,k-1\}$, $|v_i\setminus
  F^+|\le\epsilon|v_i|/2$, otherwise the phase would have terminated
  in round $i<k$.  Stated another way,
  \[
    |v_i|\le \frac{|v_i\cap F^+|}{1-\epsilon/2} \le (1+\epsilon)|v_i\cap F^+| \enspace .
  \]

  Let $\hat{v}_k=\{z\in v_k:z< y_p\}$.  Observe that $\hat{v}_k$ contains
  at most $\epsilon|v_k|/2$ vertices that are not in $F^+$, otherwise
  $v_k'\cap \hat{v}_k$ is non-empty, contradicting the minimality of $y_p$. 
  This means
  \[    |\hat{v}_k| \le |F^+\cap \hat{v}_k| + \epsilon|v_k|/2 \enspace .
  \]
  Now we have
  \begin{align*}
    y_p - x_p - 1 & = \sum_{i=0}^{k-1} |v_i| + |\hat{v}_k|  \\
    & \le (1+\epsilon)\sum_{i=0}^{k-1} |v_i\cap F^+| + |\hat{v}_k| \\
    & \le (1+\epsilon)\sum_{i=0}^{k-1} |v_i\cap F^+| + |\hat{v}_k\cap F^+| + \epsilon|v_k|/2 \\ 
    & \le  (1+\epsilon)\sum_{i=0}^{k-1}|v_i\cap F^+| + |\hat{v}_k\cap F^+| + 2\epsilon|v_{k-1}| \\ 
    & \le  (1+\epsilon)\sum_{i=0}^{k-1}|v_i\cap F^+| + |\hat{v}_k\cap F^+| + (2\epsilon+2\epsilon^2)|v_{k-1}\cap F^+| \\ 
    & \le  (1+3\epsilon+\epsilon^2)\sum_{i=0}^{k-1}|v_i\cap F^+| + |\hat{v}_k\cap F^+| \\
    & <  (1+4\epsilon)\sum_{i=0}^{k-1}|v_i\cap F^+| + |\hat{v}_k\cap F^+|  \\
    & =  (1+4\epsilon)|\{z\in F^+: x_p< z< y_p\}|
  \end{align*}

  Suppose the procedure terminates after $s-1$ phases, so that $x_1<\cdots<x_{s-1}$ are the vertices that begin each phase.  For convenience we define $x_s=y_{s-1}$, $x_0=-1$ and $x_{s+1}=n$.

  For each $p\in\{0,\ldots,s\}$, let $F_p^+=\{z\in F^+: x_p <
  x < x_{p+1}\}$.  Since none of $x_0,\ldots,x_{s+1}$ are in $F^+$,
  $F_0^+,\ldots,F_{s}^+$ is a partition of $F^+$.  

 For each $p\in\{1,\ldots,s\}$, the argument above shows that $(1+4\epsilon)|F_p^+|\ge x_{p+1}-x_p-1$.
  Furthermore, the minimality of $x_1$ ensures that $(1+4\epsilon)|F_0^+|\ge |F_0^+|=x_1=x_1-x_0-1$.  
   Therefore,
  \begin{align*}
    (1+4\epsilon)|F^+| & = (1+4\epsilon)\sum_{p=0}^s |F^+_p| \\
    & \ge \sum_{p=0}^s (x_{p+1}-x_p-1) \\
    & = n-s \enspace .
  \end{align*}
  so 
  \begin{align*}
    s & \ge n-(1+4\epsilon)|F^+| \\
    & \ge n-(1+4\epsilon)(2+4\epsilon)|F|\\
    & =n-(2+O(\epsilon))|F| \enspace . 
  \end{align*}
  Since $s$ is a lower bound on the size of the monotone path in $G-S$
  constructed by this procedure, this completes the proof.
\end{proof}


\section{Achieving $1+\epsilon$}


Next, we show how to generalize our construction so that, for any
$F\subset\{1,\ldots,n\}$, $G-F$ contains a monotone path of size
$n-(1+O(\epsilon))|F|$. To achieve this, we add additional exploder
graphs.  For each node $u$ of $T$, and each $i\in\{1,\ldots,q\}$ such
that $r^{(i)}(u)$ exists, $G$ contains the edges of an exploder graph for
the pair $(u,r^{(i)}(u))$.  Observe that this construction, with $q=1$,
corresponds exactly to the graph $G$ described in the previous section.
In this generalized construction, the number of edges is $O(q\epsilon^{-2}
n\log n)$ edges.

Given a set $F\subseteq V$, we define $F^+$ as follows: The vertices in
each node $u$ of $T$ are all included in $F^+$ if
\begin{enumerate}
   \item $u$ is dense with respect to $F$; or
   \item $r^{(i)}(u)$ exists and is dense with respect to 
   $F$ for all $i\in\{1,\ldots,q\}$.
\end{enumerate}
For any maximal node $u$ that is not dense but satisfies Condition~2
above, the nodes $r^{(1)}(u),\ldots,r^{(q)}(u)$ all exist and are dense.
It follows that
\begin{align*} 
  |F^+| & \le \left(\frac{1+1/q}{1-\epsilon}\right)|F| \\
  & \le (1+1/q)(1+2\epsilon)|F| &\text{(for any $\epsilon \le 1/2$)} \\
  & \le (1+4\epsilon)|F| &\text{(for $q=\lceil 1/\epsilon\rceil$).}
\end{align*}

\begin{clm}\clmlabel{one-plus-epsilon}
  For any $F\subseteq V(G)$, $G-F$ contains a monotone path of size
  $n-(1+O(\epsilon))|F|$.
\end{clm}

\begin{proof}
   This proof is similar to the proof of \clmref{two-plus-epsilon}
   so we only focus on the differences.

   The path is constructed in phases and each phase is made up of
   several rounds.  Phase $p$ begins at a vertex $x_p$ and ends when
   we find a monotone path from $x_p$ to some vertex $y_{p}\not\in
   F^+$. During each round, we consider the unique node $a_i$ of $T$
   that has size $2^i$ and that contains $x_p$. For Round~0 we begin at
   the unique leaf $\{x_p\}$ of $T$ that contains $x_p$.

   In addition to $a_i$ we will consider the nodes
   $v_{i,1},\ldots,v_{i,d_i}$ where $v_{i,j}=r^{(j)}(a_i)$ and $d_i$
   is the smallest integer such that $r^{(d_i)}(a_i)$ is sparse.  If no
   such integer exists, the procedure for finding the monotone path has
   completed and finishes at $x_p$.
   To remain consistent with notation in 
   the proof of \clmref{two-plus-epsilon}, we define $u_i=v_{i,d_i}$.

   For each round $i$, we will maintain the following \emph{witness
   invariant}: Each node $v_{i,j}$, $j\in\{1,\ldots,d_i\}$ has a subset
	$v_{i,j}'\subseteq v_{i,j}$ that is a witness set for $(x_p,v_{i,j})$.
  
   To proceed from round $i$ to round $i+1$ we consider the first of following cases that is applicable:
 
   \begin{enumerate}
      \item $u_i'\setminus F^+$ is non-empty. In this case,
	we take $y_p = \min(u_i'\setminus F^+)$. We
        have found a monotone path from $x_p$ to $y_{p}$ and we terminate
        the phase.  The next phase begins at $x_{p+1}=y_p$.

      \item Each of $r^{(1)}(a_{i+1}),\ldots,r^{(q)}(a_{i+1})$ is
        non-existent or dense.  In this case, the procedure terminates.
        This is the final phase and the path we have constructed ends at $x_p$.

       \item $u_i$ is the right child of $u_{i+1}$.
   This case is easy. For each node
   $v\in\{v_{i+1,1},\ldots,v_{i+1,d_{i+1}}\}$, the two children
   of $v$ are contained in $\{v_{i,1},\ldots,v_{i,d_i}\}$.
   For each such node $v$, the witness invariant ensures that
   $v$'s two children each have witness sets.  The union of these
   two witness sets is a witness set for $v$.  Thus we are
   able proceed from round $i$ to round $i+1$ while maintaining
   the witness invariant.

   \item $u_i$ is the left child of $u_{i+1}$. Let $w_i=r(u_i)$
   be the right child of $u_{i+1}$. Then $G$ contains the exploder
   graph $B_{u_i,w_i}$.  Since $u_i$ is sparse and has a witness set
   $u_i'$, $N_{B_{u_i,w_i}}(u_i')\subset w_i$ is a witness set for
   $w_i$.  Now we proceed as in the previous case since, for each node
   $v\in\{v_{i+1,1},\ldots,v_{i+1,d_{i+1}}\}$, the two children of $v$
   are contained in $\{v_{i,1},\ldots,v_{i,d_i}\}\cup\{w_i\}$.

   \item Otherwise, $u_i$ is not a child of $u_{i+1}$. In this case,
   select $j\in\{1,\ldots,d_{i+1}-1\}$ so that $v_{i+1,j}$ is the parent
   of $u_i$.  By using one of the two preceding arguments, we conclude
   that each of $v_{i,1},\ldots,v_{i,j}$ has a witness set.  All that
   remains is to show the existence of a witness set for each
   $v\in\{v_{i,j+1},\ldots,v_{i+1,d_{i+1}}\}$.	To see
   why this is the case, observe that the witness set
   $u_i'$ is a subset of $v_{i,j}$ of size at least
   $\epsilon|u_i|/2\ge \epsilon|v_{i,j}|/4$.   Now,
   $G$ contains the exploder graph $B_{v_{i,j},v}$.
   Therefore, by \obsref{extend}, $N_{B_{v_{i,j},v}}(u_i')$
   is a witness set for $v$.  Thus, we are able to maintain
   the witness invariant when proceeding from round $i$
   to round $i+1$.
   \end{enumerate}
         
   This completes the description of how to construct the path
   from $x_p$ to $y_p=x_{p+1}$ in phase $p$ of the procedure.
   The remainder of this proof proceeds almost exactly as the proof
   of \clmref{two-plus-epsilon}.  In particular, the same arguments
   used in the proof of \clmref{two-plus-epsilon} show that there are
   nodes $v_1,\ldots,v_{k}$ and a partial node $\hat{v}_k=\{z\in v_k: z<
   y_p\}$ such that (i)~$v_1,\ldots,v_{r-1},\hat{v}_k$ is a partition
   of $\{x_p+1,\ldots,y_p-1\}$; (ii) for each $i\in\{1,\ldots,r-1\}$,
   $|v_{i+1}|\le 4|v_i|$ and $|v_i\cap F^+|\ge (1-O(\epsilon))|v_i|$; and
   (iii)~$|\hat{v}_k\setminus F^+|\le\epsilon|v_k|/2$.  This establishes
   that $y_p-x_p-1 \le (1-O(\epsilon))|\{z\in F^+:x_p < z<y_p\}|$.
   At this point (after taking special care of the first and final phase),
   the same double-counting argument establishes that the number, $s$, of
   phases is at least $n-(1+O(\epsilon))|F^+|=n-(1+O(\epsilon))|F|$. Since
   the monotone path we construct contains all the vertices
   $x_1,\ldots,x_s$, this completes the proof.
\end{proof}

\section{Remarks}

\clmref{one-plus-epsilon} and the fact that the graph $G$
in \secref{one-plus-epsilon} has $O(q\epsilon^-2 n\log
n)=O(\epsilon^{-3}n\log n)$ completes the proof of \thmref{main}.

Except possibly for the dependence on $\epsilon$, this work settles
the question of $f(k)$-robust spanners in 1-dimension. The obvious
extension of this work is to 2-dimensions.  There, larger gaps remain
between the upper and lower bounds.  Indeed, for point sets in $\R^2$,
no non-trivial constructions of $f(k)$-robust $t$-spanners exist for any
constant $t>1$ and any $f(k)\in o(k^2)$.  We conjecture that, for every
constant $t>1$, every $n$-point set $V\subset\R^2$ has an $O(k)$-robust
$t$-spanner with $O(n\log n)$ edges.  More specifically, we conjecture
that if one considers a semi-separated pair decomposition $\{(A_i,B_i):
i\in\{1,\ldots,r\}\}$ \cite{abam.deberg.ea:region,abam.carmi.ea:on,abam.deberg.ea:geometric} with an appropriate separation parameter $s$
(that depends on $t$) and constructs a graph $G$ that contains an
$\epsilon$-exploder for each pair $(A_i,B_i)$ then the resulting graph
is an $O(k)$-robust $t$-spanner.

%\begin{proof}
%  Let $v_0,\ldots,v_{\log_2 n}$ be the path from the leaf of $T$
%  containing $x$ to the root of $T$.  Let $v_t$ be the last node on
%  this path such that at least one of $r^{(1)}(v_t),\ldots,r^{(q)}(v_t)$
%  is defined and is sparse.  Observe that every vertex $y > \max(v_t)$
%  is contained in $F^*$.
%
%  We claim that for every $j\in\{1,\ldots,t\}$ and every node $w$ such
%  that $w=r^{(i)}(v_j)$, for some $i\in\{1,\ldots,q\}$, $G-F$ contains a
%  monotone path from $x$ to every vertex $y\in w\setminus F$.  We prove
%  this by claim by induction on $j$.  When $j=0$, $w = r^{(i)}(v_0)$
%  for some $i\in\{1,\ldots,q\}$, so $G$ contains an exploder graph for
%  the pair $(v_0,w)$ and this graph contains the single edge joining $x$
%  to the unique vertex in $y\in w$.
%
%  Now we wish to prove this claim for any $w\in\{r^{(i)}(v_j): i\in\{1,\ldots,q\}\}$.  First observe that, for any $i\in\{1,\ldots,\lfloor(q-1)/2\rfloor\}$, the two children of $r^{(i)}(v_j)$ are contained in 
%  $\{r^{(i)}(v_{j-1}): i\in\{1,\ldots,q\}\}$. Therefore, by induction on $j$ there is a monotone path from $x$ to every vertex of $r^{(i)}(v_j)$ for every $i\in\{1,\ldots,\lfloor (q-1)/2\rfloor\}$. From this point on we therefore assume that $w=r^{(i)}(v_j)$ for some $i \ge $ TODO: Continue here!
%  
%  node $w$
%
%
%  For each $j\in\{1,\ldots,t\}$, let $v_j'$ denote the first node among
%  $r^{(1)}(v_j),\ldots,r^{q}(v_j)$ that is sparse.  To prove our claim
%  for $j>1$, consider the node $v_{j-1}'$.  From the definition of $F^*$
%  (and $t$) the node $v_{j-1}'$ exists.  Let $u$ be the parent
%  of $v_{j-1}'$.
%  Below, we will argue
%  that $w=u$ or that $w=r^{(i)}(u)$ for some $i\in\{1,\ldots,q\}$.
%  \begin{itemize}
%    \item In the case that $w=u$, the node $u$ has two children $v_{j-1}'$
%    and $v$.  If $v_{j-1}'$ is the right child of $u$ then, by induction,
%    there is a monotone path from $x$ to every vertex in $v\setminus F$
%    and a monotone path from $x$ to every vertex in $v_{j-1}'\setminus F$
%    and we are done.  If $v_{j-1}'$ is the left child of $u$, then there
%    is a monotone path from $x$ to every vertex in $x'\in v_{j-1}'$.
%    Now, $v_{j-1}'\cap F$ is a subset of $v_{j-1}'$ of size at least
%    $\epsilon |v_{j-1}'|$ and $G$ contains an exploder graph for the
%    pair $(v_{j-1}',v)$.  Therefore, for every $y\in v$ there is
%    vertex in $x'\in v_{j-1}'$ adjacent to $y$ and therefore there is
%    a monotone path from $x$ to $x'$ to $y$, as required.
%
%    \item In the case that $w=r^{(i)}(u)$ for some $i\in\{1,\ldots,q\}$,
%    $v_{j-1}'\setminus F$ is a subset of $u$ of size at least
%    $\epsilon|v_{j-1}'| \ge \epsilon|u|/2$.  Now $G$ contains an exploder
%    graph for the pair $(u,w)$.  Therefore every vertex $y\in w$ is
%    adjacent to some vertex $x'\in v_{j-1}'\setminus F$.  By induction there
%    is a monotone path from $x$ to $x'$ and the edge $x'y$ extends this
%    to a monotone path from $x$ to $y$, as required.
%  \end{itemize}
%
%  Thus, all that remains is to show that $w=u$ or $w=r^{(i)}(u)$ for
%  some $i\in\{1,\ldots,q\}$.  Let $\tau\in\{0,\ldots,q-1\}$ be such that
%  $v_{j-1}'=r^{(\tau+1)}(v_j)$.  From the definition of $v_{j-1}'$, we
%  know that $r^{(1)}(v_{j-1}),\ldots,r^{(\tau)}(v_{j-1})$ are all dense
%  nodes.  Any node of $T$ with 2 dense children is itself dense. Therefore
%  $r^{(1)}(v_j),\ldots,r^{(\lfloor\tau/2\rfloor)}(v_j)$ are all dense.
%  Therefore, $w=r^{(\kappa)}(v_j)$ for some $\kappa >
%  \lfloor\tau/2\rfloor$.  On the other hand, $u=r^{(\gamma)}(v_j)$ for
%  some $\gamma\in\{\lfloor\tau/2\rfloor,\lfloor\tau/2\rfloor+1\}$.
%  In either case $\kappa \ge \gamma$ so $w=u$ or
%  $w=r^{(\kappa-\gamma)}(u)$, as required.
%\end{proof}
%
%\section{Crap}
%
%Then $\|vw\|_{G-F}=\infty$ so $G-F$ is a not a $t$-spanner of $V\setminus F=V\setminus F^*$ for any $t<\infty$.}
%A slightly more sophisticated argument shows that any $(1+1/d)k$-robust $t$-spanner must have $\Omega(dn)$ edges.\footnote{Proof: If $G$ has $
%
%rules out the existence
%of $(k+o(k))$-robust spanners having a subquadratic number of edges.
%
%
%

\bibliographystyle{plain}
\bibliography{robust2}

\end{document}









